\newglossaryentry{Actuator}
{
	name = {Actuator},
	description = {
		 In a \gls{closedLoop} control system, that part of the final control element that translates the controller output into an action by the control device.}
}

\newglossaryentry{conSys}
{
	name = {Control System},
	description = {
		A control system is an interconnection of components forming a system configuration that will provide a desired system response.
	}
}
\newglossaryentry{openLoop}
{
	name = {open-loop control system},
	description = {
		An open-loop control system utilizes an actuating device to control the process
		directly without using feedback.
	}
}
\newglossaryentry{closedLoop}{
	name={closed-loop control system},
	description={A closed-loop control system uses a measurement of the output and feedback of this signal to compare it with the desired output (reference or command).
	}
}

\newglossaryentry{SISO}
{
	name = {Single Input Single Output},
	description = {
		In control engineering, a single-input and single-output (SISO) system is a simple single variable control system with one input and one output. SISO systems are typically less complex than multiple-input multiple-output (MIMO) systems. Frequency domain techniques for analysis and controller design dominate SISO control system theory. }
}

\newglossaryentry{MIMO}
{
	name = {Multiple Input Multiple Output},
	description = {
		In control engineering, systems with more than one input and/or more than one output are known as Multi-Input Multi-Output systems, or they are frequently known by the abbreviation MIMO. MIMO systems that are lumped and linear can be described easily with state-space equations.}
}
\newglossaryentry{LTI}
{
	name = {linear time-invariant},
	description = {
		\textbf{Linear time-invariant systems} (LTI systems) are a class of systems used in signals and systems that are both linear and time-invariant. Linear systems are systems whose outputs for a linear combination of inputs are the same as a linear combination of individual responses to those inputs. Time-invariant systems are systems where the output does not depend on when an input was applied. 
	}
}

\newglossaryentry{cornFreq}
{
	name = {Corner Frequency},
	description = {
	 For first order systems, the corner frequency is the frequency where the magnitude starts to roll-off (3dB below the steady-state gain) and the phase shift is -45 degrees. Also: corner frequency = 1/(time constant) rads/s.
	}
}

\newglossaryentry{controlGain}
{
	name = {Controller Gain},
	description = {
		This is another term for the "P" part of the PID controller. The more gain a controller has the faster and potentially more oscillatory the loop response will be.
	}
}

\newglossaryentry{critDamp}
{
	name = {critically-damped},
	description = {
		A linear system that has the fastest response without any overshoot is said to be critically damped.
	}
}

\newglossaryentry{bodeDia}
{
	name = {bode diagram},
	description = {
		Graphical display of the frequency response with magnitude and phase both plotted against frequency on the horizontal axis.
	}
}
