\begin{center}
	\hrule
	\vspace{.4cm}
	{\textbf { \large ELEC 460 --- Control Theory II}}
\end{center}
{\textbf{Name:}\ David Li \hspace{\fill} \textbf{Student Number:}} \ V00818631  \\
{\textbf{Due Date:} Wednesday, March 14, 11:30 AM \hspace{\fill} \textbf{Assignment:} Number 7} \\
	\hrule
	
	
\subsubsection*{B-5-18}
Consider the discrete-time state equation

\[
\begin{bmatrix}
x_1(k+1) \\
x_2(k+1)
\end{bmatrix} = \begin{bmatrix}
0     &  1 \\
-0.24 & -1 
\end{bmatrix}\begin{bmatrix}
x_1(k) \\
x_2(k)
\end{bmatrix}
\]
Obtain the state transition matrix $\Psi(k)$

Since $\displaystyle \psi(k)=G^k= \mathcal{Z}^{-1} \left\{z[zI-G]^{-1}\right\}$, and $A=\begin{bmatrix}
a     &  b \\
c &  d 
\end{bmatrix}^{-1}= \frac{1}{ad-bc}\begin{bmatrix}
d     &  -b \\
-c &  a 
\end{bmatrix}$

\begin{align*}
& \mathcal{Z}^{-1} \left\{z[zI-G]^{-1}\right\} = z \begin{bmatrix}
z     &  -1 \\
0.24 & z+1
\end{bmatrix}^{-1}= \mathcal{Z}^{-1} \left\{
\frac{z}{z^2+z-(-1)(0.24)}\begin{bmatrix}
z+1     &  1 \\
-0.24 & z
\end{bmatrix}  \right\} \\
& \frac{F(z)}{z}= %\mathcal{Z}^{-1} \left\{
\frac{1}{(z+0.4)(z+0.6)}\begin{bmatrix}
z+1     &  1 \\
-0.24 & z
\end{bmatrix} %\right\} = % \mathcal{Z}^{-1} \left\{
 = \left(\begin{bmatrix} \frac{15}{5\,z+2}-\frac{10}{5\,z+3} & \frac{25}{5\,z+2}-\frac{25}{5\,z+3}\\ \frac{6}{5\,z+3}-\frac{6}{5\,z+2} & \frac{15}{5\,z+3}-\frac{10}{5\,z+2} \end{bmatrix}\right) \\ %\right\} 
 & \psi(k)=\mathcal{Z}^{-1} \left\{z[zI-G]^{-1}\right\} = G^k = %\begin{bmatrix} z\,\left(\frac{15}{5\,z+2}-\frac{10}{5\,z+3}\right) & z\,\left(\frac{25}{5\,z+2}-\frac{25}{5\,z+3}\right)\\ -z\,\left(\frac{6}{5\,z+2}-\frac{6}{5\,z+3}\right) & -z\,\left(\frac{10}{5\,z+2}-\frac{15}{5\,z+3}\right) \end{bmatrix} = 
 \begin{bmatrix} 3.0\,{\left(-0.4\right)}^k-2.0\,{\left(-0.6\right)}^k & 5.0\,{\left(-0.4\right)}^k-5.0\,{\left(-0.6\right)}^k\\ 1.2\,{\left(-0.6\right)}^k-1.2\,{\left(-0.4\right)}^k & 3.0\,{\left(-0.6\right)}^k-2.0\,{\left(-0.4\right)}^k\end{bmatrix}
\end{align*}

% Use LIAPUNOV Stability analysis
\subsubsection*{B-5-22}
Determine the stability of the origin of the following discrete-time system:
\[
\begin{bmatrix}
x_1(k+1) \\
x_2(k+1) \\
x_3(k+1)
\end{bmatrix} = \begin{bmatrix}
1  & 3  &  0 \\
-3 & -2 & -3 \\
1  & 0  &  0
\end{bmatrix}\begin{bmatrix}
x_1(k) \\
x_2(k) \\
x_3(k)
\end{bmatrix}
\]

\begin{align*}
& \det(zI-G) = \det \left(\begin{bmatrix}
z-1 &   -3 & 0  \\
3  & z+2 & 3 \\
-1   &  0  & z 
\end{bmatrix} \right) = (z-1)(z+2)z-(3)(-3z+3) = z^3+z^2+7z+9
\end{align*}
Using the characteristic polynomial and the jury-marden stability criterion.

$P(z)=a_0z^3+a_1z^2+a_2z+a_3=z^3+z^2+7z+9$, and since $|a_3| < a_0 \rightarrow 9 < 1$, is not true, the origin of the system is unstable.
\subsubsection*{B-6-1}
Consider the system defined by
\begin{align*}
& \begin{bmatrix}
x_1(k+1) \\
x_2(k+1)
\end{bmatrix}= \begin{bmatrix}
a  &  b  \\
c  &  d
\end{bmatrix}\begin{bmatrix}
x_1(k) \\
x_2(k)
\end{bmatrix}+\begin{bmatrix}
1 \\
1
\end{bmatrix}u(k) \\
& y(k) = [1 \quad 0] \begin{bmatrix}
x_1(k) \\
x_2(k)
\end{bmatrix}
\end{align*}
Determine the conditions on a,b,c, and d for complete state controllability and complete observability. $\displaystyle G = \begin{bmatrix} a & b\\ c & d \end{bmatrix}, \quad H =\begin{bmatrix} 1 \\ 1 \end{bmatrix}$

%\paragraph{State Controllability}
\noindent $\text{rank}\begin{bmatrix}
H & GH
\end{bmatrix}=\left(\begin{array}{cc} 1 & a+b\\ 1 & c+d \end{array}\right)=n=2$ \hfill \break 

\paragraph{Complete State Controllability}, $a+b \neq c+d$ \hfill \break 

Since row rank equals column rank, the rank of a matrix is the same of its transpose, $\text{rank}\left(\begin{array}{cc} 1 & 0\\ a & b \end{array}\right) =n =2$, the condition is $b \neq 0$

\subsubsection*{B-6-5}

For the system defined by
\begin{align*}
& \begin{bmatrix}
x_1(k+1) \\
x_2(k+1)
\end{bmatrix}= \begin{bmatrix}
0     &  1  \\
-0.16 & -1
\end{bmatrix}\begin{bmatrix}
x_1(k) \\
x_2(k)
\end{bmatrix}+\begin{bmatrix}
0 \\
1
\end{bmatrix}u(k) \\
& y(k) = [1 \quad 0] \begin{bmatrix}
x_1(k) \\
x_2(k)
\end{bmatrix}
\end{align*}
assume that the following outputs are observed:
$y(0)=1$ and $y(1)=2$
The control signals give are
$u(0)=2$, $u(1)=-1$.
Determine the initial state x(0). Also determine states x(1) and x(2).

Since $y(0)=x_1(0)=1 \quad y(1)=x_1(1)=2$, $x(0)= \begin{bmatrix}
1 \\ 2
\end{bmatrix}$
State x(1) is:
	\[
	\begin{bmatrix}
	x_1(1) \\
	x_2(1)
	\end{bmatrix} = 	\begin{bmatrix}
	0     &  1 \\
	-0.16 & -1
	\end{bmatrix}	\begin{bmatrix}
	1 \\
	2
	\end{bmatrix}+	\begin{bmatrix}
	0 \\
	1
	\end{bmatrix}[2]=	\begin{bmatrix}
	2 \\
	-0.16
	\end{bmatrix}
	\]
and state x(2) is
	\[
\begin{bmatrix}
x_1(2) \\
x_2(2)
\end{bmatrix} = 	\begin{bmatrix}
0     &  1 \\
-0.16 & -1
\end{bmatrix}	\begin{bmatrix}
2 \\
-0.16
\end{bmatrix}+	\begin{bmatrix}
0 \\
1
\end{bmatrix}[-1]=	\begin{bmatrix}
-0.16 \\
-1.16
\end{bmatrix}
\]