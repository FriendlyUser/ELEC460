
\begin{center}
	\hrule
	\vspace{.4cm}
	{\textbf { \large ELEC 460 --- Control Theory II}}
\end{center}
{\textbf{Name:}\ David Li \hspace{\fill} \textbf{Student Number:} \ V00818631  \\}
{\textbf{Due Date:} April 3,2018 \hspace{\fill} \textbf{Assignment}  9}\\
\hrule
\subsubsection*{Problem B-6-17}
\vspace*{-0.5cm}
 Figure 6-24 shows a servo system where the integral controller has a time delay of one sampling period. (Compare this system with the servo system shown in Figure 6-18)
\begin{figure}[H]
	\centering
	\includegraphics[width=1\linewidth]{OgataB6-17.pdf}
	\caption{Ogata B-6-17}
	\label{fig:ogatab6-17}
\end{figure}

\vspace*{-1cm}
\begin{align*}
& v(k+1) = v(k) +r(k) - y(k) \quad x(k+1) = 0.5x(k)+u(k) \\
& y(k) = x(k) \quad  u(k) = -K_2 x(k) +K_1 v(k) \quad K_1v(k) =u(k)+K_2x(k) \\
& u(k+1) = -K_2 x(k+1)+K_1 v(k+1) = K_2(-0.5x(k)+u(k)) + K_1(v(k) +r(k) - y(k)) \\
& u(k+1) = -K_2 (-0.5x(k)+u(k)) + (u(k)+K_2x(k)) +K_1 r(k) - K_1x(k) \\
& u(k+1) = (0.5K_2-K_1)x(k)+(1-K-2)u(k)+K_1(r(k)) 
\end{align*}
Creating a state-space representation
\vspace*{-0.25cm}
\begin{align*}
& \begin{bmatrix}
x(k+1) \\
u(k+1)
\end{bmatrix} = \begin{bmatrix}
0.5       &  1 \\
0.5K_2-K_1 & 1-K_2
\end{bmatrix}\begin{bmatrix}
x(k) \\
u(k)
\end{bmatrix} + \begin{bmatrix}
0  \\
K_1
\end{bmatrix} r(k) \\
& y(k) = \begin{bmatrix}
1 & 0
\end{bmatrix}\begin{bmatrix}
x(k) \\
u(k)
\end{bmatrix} \\
& x_e(k) = x(k) - x(\infty) \\
& u_e(k) = u(k) - u(\infty) \\
& \begin{bmatrix}
x_e(k+1) \\
u_e(k+1)
\end{bmatrix}\begin{bmatrix}
0.5        & 1    \\
0.5K_2-K_1 & 1-K_2
\end{bmatrix}\begin{bmatrix}
x_e(k) \\
u_e(k)
\end{bmatrix} \\
& (Iz-G)=\begin{bmatrix}
z-0.5        & -1  \\
-0.5K_2 +K_1 & z-1+K_2
\end{bmatrix}=z^2+(K_2-1.5)z+0.5-K_2+K_1=0
\end{align*}
The desired characteristic equation $P(z)=0$, $z^2=0$, choose $K_2=1.5$, $K_1=1$. 
\begin{align*}
& \begin{bmatrix}
x(k+1) \\
u(k+1)
\end{bmatrix}=\begin{bmatrix}
0.5       &  1 \\
-0.25     & -0.5
\end{bmatrix}\begin{bmatrix}
x(k) \\      
u(k) 
\end{bmatrix} \\
& \text{Assume that the initial state is:} \begin{bmatrix}
x(0) \\      
u(0) 
\end{bmatrix} = \begin{bmatrix}
a \\      
b 
\end{bmatrix} \\
& \begin{bmatrix}
x(1) \\      
u(1) 
\end{bmatrix} = \begin{bmatrix}
0.5       &  1 \\
-0.25     & -0.5
\end{bmatrix}\begin{bmatrix}
a \\      
b 
\end{bmatrix}+ \begin{bmatrix}
0  \\
1
\end{bmatrix} = \begin{bmatrix}
0.5a +b \\
-0.25a -0.5b+1
\end{bmatrix} \\
& \begin{bmatrix}
x(2) \\      
u(2) 
\end{bmatrix} = \begin{bmatrix}
0.5       &  1 \\
-0.25     & -0.5
\end{bmatrix} \begin{bmatrix}
0.5a +b \\
-0.25a -0.5b+1
\end{bmatrix} + \begin{bmatrix}
0  \\
1
\end{bmatrix} =\begin{bmatrix}
0  \\
0.5
\end{bmatrix} \\
& \begin{bmatrix}
 x(k) \\
 u(k)
\end{bmatrix}=\begin{bmatrix}
1  \\
0.5
\end{bmatrix}, \quad \text{for k}=3,4,5 \cdots 
\end{align*}
 $y(0)=x(0)=a \quad y(1)=x(1)=0.5a+b \quad y(k)=x(k)=1 \quad  \text{for k}=2,3,4 \cdots $
 
 \begin{figure}[H]
 	\centering
 	\includegraphics[width=0.5\linewidth]{stepResponsePlot.pdf}
 	\caption{Step Response}
 	%\label{fig:ogatab6-17}
 \end{figure}

\subsubsection*{Problem B-6-13}
Consider the system
\begin{align*}
& x(k+1)=Gx(k)+Hu(k) \\
& y(k) = Cx(k)
\end{align*}
where 
\begin{align*}
& x(k) = \text{state vector (2-vector)} \\
& u(k) = \text{control vector (scalar)} \\
& y(k) =  \text{output signal (scalar)}
\end{align*}
and 
\[
G = \begin{bmatrix}
0     & 1 \\
-0.16 & -1
\end{bmatrix}, \qquad
H = \begin{bmatrix}
0    \\
1
\end{bmatrix}, \qquad 
C = \begin{bmatrix}
0  & 1
\end{bmatrix}
\]
\textbf{Design a full order prediction observer with deadbeat response for the system of Question B-6-13 in the textbook.}

Prediction (full order) observer where the estimate $x(k+1)$ is obtained based on measurements of up to $y(k)$.

\vspace*{-1cm}

\begin{align*}
& \tilde{x}(k+1)=G\tilde{x}(k)+Hu(k)+k_e[y(k)-\tilde{y}(k)] \\
& \tilde{y}(k+1) = c\tilde{x}(k+1) 
\end{align*}
$k_e$ for this observer can be obtained using the inverse observability matrix, O

\[
k_e = O^{-1} \tilde{A}^{-1}(\alpha-a)^T
\]
where $\alpha=0$ since the observer is required to have a deadbeat response. Further,
\begin{align*}
& \det(zI-G)=z^2+z+0.16 \qquad a =[1 \ \ 0.16] \\
& \tilde{A} = \begin{bmatrix}
1 & 0 \\
1 & 1 
\end{bmatrix} \\
& O = \begin{bmatrix}
c \\
cG
\end{bmatrix} = \begin{bmatrix}
1     & 1 \\
-0.16 & 0
\end{bmatrix} \\
& k_e = \begin{bmatrix}
-5.25 \\ 4.25
\end{bmatrix}
\end{align*}
which leads to:
\[
\tilde{x}(k+1)=  \begin{bmatrix}
1     & 1 \\
-0.16 & 0
\end{bmatrix}\tilde{x}(k) + \begin{bmatrix}
0 \\ 1
\end{bmatrix} u(k) +  \begin{bmatrix}
-5.25 \\ 4.25
\end{bmatrix} (y(k)- \tilde{y}(k))
\]

