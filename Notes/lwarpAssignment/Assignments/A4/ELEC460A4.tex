
\begin{center}
	\hrule
	\vspace{.4cm}
	{\textbf { \large ELEC 460 --- Control Theory II}}
\end{center}
{\textbf{Name:}\ David Li \hspace{\fill} \textbf{Student Number:} \ V00818631}  \\
{\textbf{Due Date:} February 20, 2018, 11:30 AM \hspace{\fill} \textbf{Assignment}  4}\\
\hrule

\subsubsection*{Problem B-3-20}
Assuming that the sampling period is 0.2 sec, and the gain constant K is unity, determine the response c(kT) for k = 0,1,2,3 and 4 when the input r(t) is a unit-step function. Also, determine the final value $c(\infty)$.
\begin{figure}[H]
	\centering
	\includegraphics[width=0.6\linewidth]{Diagrams/samplerBlock.pdf}
	\caption{Block Diagram for B-3-20}
	\label{fig:samplerblock}
\end{figure}

Table 3-1 from the textbook %\cite{ogataDiscrete}
 $C(z)=\frac{[GR(z)]}{1+GH(z)}$, where $GH(z) = \left[G(s)H(s)\right]^\ast=[1-z^{-1}] \mathcal{Z} \left\{\frac{1}{s(s+1)}\right\}$
\begin{align*}
&  \mathcal{Z} \left\{\frac{1}{s(s+1)}\right\} = \mathcal{Z} \left\{\frac{1}{s}+\frac{-1}{s+1}\right\} = \frac{z}{z-1}-\frac{z}{z-e^{-(0.2)(1)}}=\frac{z}{(z-1)(z-e^{-0.2})} \\
& GH(z) = \left[G(s)H(s)\right]^\ast=[1-z^{-1}] \mathcal{Z} \left\{\frac{1}{s(s+1)}\right\} 
= 1-\frac{z-1}{z-e^{-0.2}}=\frac{1-e^{-0.2}}{z-e^{-0.2}}=\frac{0.181269}{z-0.818731} \\
& R(z)G(z)= \mathcal{Z} \left\{\frac{1}{s}\frac{1}{(s+1)}\right\} = \frac{0.181269z}{(z-1)(z-0.818731)} \\
& C(z) =\frac{\frac{0.181269z}{(z-1)(z-0.818731)}}{1+\frac{0.181269}{z-0.818731}}= \frac{\frac{0.181269z}{(z-1)(z-0.818731)}}{\frac{z-0.818731+0.181269}{z-0.818731}}= \frac{0.1813z}{(z-1)(z-0.8187)} \times \frac{z-0.819}{z-0.637}=\frac{0.1813z}{(z-1)(z-0.6375)} 
 %\frac{z^2}{(z-1)(z-0.818731)}
\end{align*}
Using final value theorem:

\[ c(\infty)=\lim\limits_{z\rightarrow 1} \left[ \left(1-z^{-1}\right)C(z)\right]=\lim\limits_{z\rightarrow 1} \left[ \left(1-z^{-1}\right)\frac{0.1813z^{-1}}{(1-z^{-1})(1-0.6375z^{-1})} \right]=0.5001379 \]
\subsubsection*{Problem B-4-4}
Consider the discrete-time closed-loop control system shown in Figure 4-13. Determine
the range of K for stability by use of the Jury stability criterion. Assuming that $T=1$.

\begin{figure}[H]
	\centering
	\includegraphics[width=1\linewidth]{Diagrams/sampler413.pdf}
	\caption*{Figure 4-13}
	\label{fig:samplerblock413}
\end{figure}

\begin{align*}
&(1-z^{-1}) \mathcal{Z} \left\{ \frac{1}{s^2}\frac{1}{s+1} \right\}= (1-z^{-1}) \mathcal{Z} \left\{ \frac{1}{s^2}+\frac{-1}{s}+\frac{1}{s+1} \right\} =  \left(\frac{Tz^{-1}}{1-z^{-1}}-1+\frac{1-z^{-1}}{1-e^{-1}z^{-1}}\right) \\
& = \frac{z^{-1}}{1-z^{-1}}-1+\frac{1-z^{-1}}{1-e^{-1}z^{-1}} =\frac{z^{-1}-e^{-1}z^{-2}
	-(1-e^{-1}z^{-1}+e^{-1}z^{-2}) +(1-2z^{-1}+z^{-2}
	)}{(1-z^{-1})(1-e^{-1}z^{-1})} \\
& = \frac{e^{-1}z^{-1}+(1-2e^{-1})z^{-2}}{(1-z^{-1})(1-e^{-1}z^{-1})} \quad G(z) = \frac{K[e^{-1}z^{1}+(1-2e^{-1})]}{(z-1)(z-e^{-1})} \quad \frac{C(s)}{R(s)}=\frac{G(z)}{1+G(z)(1)} \\
& P(z)= a_0z^2+a_1z+a_2=z^2-(1.3679-0.3679K)z+(0.3679+0.2642K) \\
& |a_2| < |a_0| \quad |0.3679+0.2642K| < |1| \rightarrow K =2.3925056 \\
& P(1) > 0 \ 1-1.3679-0.3679K + 0.3679+0.2642K > 0 \rightarrow 0.6321K > 0 \\
& P(-1) > 0 \ 1+(1.3679-0.3679K)+0.3679+0.2642K > 0 \quad 2.7358-0.1037K > 0 \ K > 26.3818
\end{align*}
For stability $0 < K < 2.3925056$.

\subsubsection*{Problem B-4-8}
Consider the digital control system shown in Figure 4-66. Plot the root loci as the gain K is varied from $0$ to $\infty$. Determine the critical value of gain K for stability. The sampling period is 0.1 sec, or $T=0.1$ What value of gain K will yield a damping ration $\zeta$ of the closed-loop poles equal to 0.5? With gain K set to yield $\zeta=0.5$, determine the damped natural frequency $\omega_d$ and the number of samples per cycle of damped sinusoidal oscillation.

\begin{figure}[H]
	\centering
	\includegraphics[width=1\linewidth]{Diagrams/B4-8.pdf}
	\caption*{Figure 4-66 with $T=0.1$}
	\label{fig:samplerblock48}
\end{figure}
\vspace*{-1.05cm}
\begin{align*}
P(z)&=(z-1)(z-0.6065)+K(z+1) \\
    s&=z^2+(K-1.6065)z+(0.6065-K) \\
    & |a_2| < |a_0| \rightarrow |0.6065+K| < 1 \quad K =0.3935 \\
    & P(1) > 0 \rightarrow 2K > 0 \\
    & P(-1) > 0 \rightarrow 3.213 > 0 \\
    & 0 < K < 0.3935
\end{align*}

\begin{figure}[H]
	\centering
	\includegraphics[width=1\linewidth]{Diagrams/B-4-8Rlocus2.png}
	\caption*{Root Locus For Figure 4-66}
	\label{fig:Rlocus}
\end{figure}

When $K=0.0646, \zeta=0.5$ with the pole at $z=0.771+j0.277=|0.8192|e^{j19.7620^\circ}$. 
\begin{align*}
z=  & |e^{-T(\zeta \omega_n)} |e^{T \omega_d} = |0.8192|e^{j19.7620^\circ} \\
& \omega_n = \frac{\ln(0.8192)}{(-0.1)(0.5)}=3.98854 = 4 \ \text{rad/s} \\
& \omega_d = \frac{(19.7629^\circ)}{0.1} = 3.449 \quad  \omega_d = \omega_n \sqrt{1-\zeta^2}=3.988 (0.866025)=3.4537 \ \text{rad/s} \\
& \text{Number of Samples per Cycle } = \frac{360^\circ}{T\omega_d}=\frac{360^\circ}{19.7629^\circ}=18.22
\end{align*}
	 %$P(z)=z^2-1.5419z+0.5419$, %$2 \zeta \omega_n =-1.5419$
%\begin{figure}[H]
%	\centering
%	\includegraphics[width=1\linewidth]{Diagrams/B4-8.pdf}
%	\caption*{Figure 4-66}
%	\label{fig:samplerblock48}
%\end{figure}


\subsubsection*{Problem B-4-12}
Design a digital proportional-plus-derivative controller for the plant whose transfer function is $1/s^2$, as shown in Figure 4-70. It is desired that the damping ratio $\zeta$ of the dominant closed-loop poles be 0.5 and the undamped natural frequency be 4 rad/sec. The sampling period is 0.1 sec, or $T=0.1$. After the controller is designed, determine the number of samples per cycle of damped sinusoidal oscillation.
\begin{figure}
	\centering
	\includegraphics[width=1\linewidth]{Diagrams/block412.pdf}
	\caption*{Figure 4-70}
	\label{fig:blockB4-12}
\end{figure}
% https://www.wolframalpha.com/input/?i=inverse+laplace+transform+of+s
\vspace*{-1.05cm}
\begin{align}
&  G_{PD}(s) = K_ds +K_p \quad G(z) = (1-z^{-1}) \mathcal{Z} \left\{\frac{1}{s^3} \right\} = \frac{(1-z^{-1})}{2} \mathcal{Z} \left\{\frac{2}{s^3} \right\} = \frac{(1-z^{-1})}{2} \frac{T^2z^{-1}(1+z^{-1})}{(1-z^{-1})^3} \\
& G_{PD}(z) =K_p+K_d(1-z^{-1})= (K_p+K_d)\frac{z-\frac{K_d}{K_p+K_d}}{z}, \quad G(z)=\frac{T^2z^{-1}(1+z^{-1})}{2(1-z^{-1})^2}=0.005\frac{(z+1)}{(z-1)^2} \\
& \omega_d = \omega_n \sqrt{1-\zeta^2} = 4 \sqrt{1-0.5^2}= 2 \sqrt{3} =3.4641 \ \text{rad/s}, \quad |z| = e^{T \zeta \omega_n} = e^{-T \zeta \omega_n} =e^{-0.2} = 0.8187 \\
& \angle z = \angle T \omega_d = 0.34641 \ \text{rad} =19.8478^\circ \quad  \text{Desired Pole } z = 0.8187 \angle 19.8478^\circ =0.7701+j0.2780 \\
& G_{PD}(z)G(z)= (K_p+K_d)\frac{z-\frac{K_d}{K_p+K_d}}{z}0.005\frac{(z+1)}{(z-1)^2}
\end{align}

Computing the angle deficiency:
% https://www.wolframalpha.com/input/?i=1%2F(-0.2299%2Bi*0.2780)%5E2
% https://www.wolframalpha.com/input/?i=atan(0.2780%2F(0.7701-x))%2Bpi%3D0.500055555*pi
%\begin{align*}
%&  \left. \phase{\frac{1}{(z-1)^2}}\right \vert_{z=0.7701+j0.2780}
%\phi_1=\phase{\frac{1}{(z-1)^2}} \bigg\vert_{z=0.7701+j0.2780} = \phase{\frac{1}{(-0.2299+j0.2780)^2}}= 100.82^\circ \quad \phi_2=\phase{z+1}\vert_{z=0.7701+j0.2780} =8.9255^\circ 
%\end{align*}
%\begin{align*}
%& \phi_3 = \phase{\frac{1}{z}} \bigg\vert_{z=0.7701+j0.2780} =-19.84918^\circ \quad  \phi=180^\circ-\phi_1 -\phi_2-\phi_3=90.10^\circ
%\end{align*}
%\begin{align*}
%&\phi = \phase{z-\frac{K_d}{K_p+K_d}}\bigg\vert_{z=0.7701+j0.2780}=90.10^\circ \rightarrow \tan^{-1} \left({\frac{0.2780}{0.7701-\frac{K_d}{K_p+K_d}}} \right) + 180^\circ= 90.10^\circ \\
%& \frac{K_d}{K_p+K_d} = 0.770149 \rightarrow G_{PD}(z)G(z)= (K_p+K_d)\frac{z-0.770149 }{z}0.005\frac{(z+1)}{(z-1)^2}
%\end{align*}
%\begin{align}
%& \frac{K_d}{K_p+K_d} = 0.770149 \\
%&\left|(K_p+K_d)\frac{z-0.770149 }{z}0.005\frac{(z+1)}{(z-1)^2} \right|\bigg\vert_{z=0.7701+j0.2780} =1
%\end{align}
Solving for equations (1) and (2),

\[ 
K_p=9.833174968 \quad  K_d=32.9474741
\]

The controller $\displaystyle G_{PD}=42.7806\frac{z-0.770149}{z} = 42.7806(1-0.770149z^{-1})$. The number of cycles per second $n=\frac{360^\circ}{19.84918^\circ}=18.13678$.


\paragraph{Matlab Code}
\begin{lstlisting}[language = Matlab,frame=single,caption={}]
%% ELEC 460 Assignment 4

%% Question 1
K =1;
Gcont = zpk([],[-1],K)
% GClosed = feedback(Gcont,1)
c2d(Gcont,0.2,'ZOH')
Gcont50 = zpk([],[-1,0],K)
c2d(Gcont50,0.2)
%% Question 2 
G2 = zpk([],[0,-1],1)
G2Discrete = c2d(G2,1,'ZOH')
syms z K
Pz = (z-1)*(z-exp(-1))+K*[z*exp(-1)+(1-2*exp(-1))]
vpa(expand(Pz),10)
%equation = [0.2642411177*K - 1.367879441*z + 0.3678794412*K*z + z^2 + 0.3678794412]
juryStab = [1 0.3679*K-1.368 0.2642*K+0.3679]
[M,L]=jury(juryStab)

%% Question 3
z = tf('z',0.1);
H = (z + 1) / ((z-1)*(z-0.6065));
kval = 0:0.005:10;
% append k =0.500
% Get more data points for k bewteen 0.591 and 0.0693
kstart = k(1:15);
kend = k(16:65);
kmid = kstart(15):0.00025:kend(1); 
% Create new gain vector
kuse = [kstart kmid kend ]
rlocus(H,kuse)
axis([-3.5 1.25 -2 2])
roots([1 -1.5419 0.5419])

%% Question 4
%
syms s 
G3 = 1/s^3
partfrac(G3,s)
complexTest =  (0.7701-1+i*0.2780)^-2
phi1 = rad2deg(angle(complexTest))

atan(0.2780/(0.7701-0.7719))+pi

%% 
% Question 4 Answering Questions
syms x Kp Kd
eqn2 =  (Kp+Kd)*abs((x-0.770149)/x * (0.005) * (x+1)/(x-1)^2)==1
eqn2 = subs(eqn2,x,0.7701+j*0.2780)
eqn1 = Kd/(Kp+Kd)==0.770149
sol =solve([eqn1, eqn2],[Kp Kd])
portSol = vpa(sol.Kp,10)
dervSol = vpa(sol.Kd,10)
\end{lstlisting}