\documentstyle[12pt]{article}
\textheight 9in
\textwidth 6.5in
\topmargin 0in
\oddsidemargin 0in
\evensidemargin -0.5in
\usepackage{html}
\begin{document}
\begin{center}
{\LARGE \bf The Z-Transform}
\end{center}

%{\large
\section*{From Discrete-Time Fourier Transform to Z-Transform}

{\bf Forward Z-Transform}

The Fourier transform of a discrete signal $x[n]$ is defined as:
\[ X(e^{j\omega})={\cal F}[x[n]]=\sum_{n=-\infty}^\infty x[n]e^{-j2\pi f n}
=\sum_{n=-\infty}^\infty x[n]e^{-j\omega n} \]
provided $x[n]$ is absolutely summable:
\[ \sum_{n=-\infty}^\infty | x[n] |<\infty \]
Obviously some signals may not satisfy this condition and their Fourier transform
do not exist. To overcome this difficulty, we can multiply the given $x[n]$
by an exponential function $e^{-\sigma n}$ so that $x[n]$ may be forced to be
summable for certain values of the real parameter $\sigma$. Now the discrete
time Fourier transform becomes:
\begin{eqnarray}
  {\cal F}[x[n]e^{-\sigma n}]&=&\sum_{n=-\infty}^\infty [x[n]e^{-\sigma n}]\;e^{-j\omega n}
  =\sum_{n=-\infty}^\infty x[n]e^{-(\sigma +j\omega) n}
  \nonumber \\
  &=&\sum_{n=-\infty}^\infty x[n]e^{-s n}=\sum_{n=-\infty}^\infty x[n] z^{-n}
  =X(z)
  \nonumber 
\end{eqnarray}
The result of this summation is a function of a complex variable defined as:
\[ z=e^s=e^{\sigma+j\omega} \]
This is the forward Z-transform of the discrete signal $x[n]$:
\[ X(z)={\cal Z}[x[n]]=\sum_{n=-\infty}^\infty x[n] z^{-n} \]

{\bf Inverse Z-Transform}

Given the Z transform $X(z)$, the original time signal can be obtained by 
the inverse Z transform, which can be derived from the corresponding Fourier 
transform. As shown above, we have:
\[ X(z)=X(e^{\sigma+j\omega})=\sum_{n=-\infty}^\infty [x[n] e^{-\sigma n}] 
e^{-j\omega n} ={\cal F}[x[n] e^{-\sigma n}]
\]
Now $x[n]e^{-\sigma n}$ can be obtained by the inverse Fourier transform:
\[ x[n] e^{-\sigma n}={\cal F}^{-1}[X(e^{\sigma+j\omega})]
=\frac{1}{2\pi}\int_0^{2\pi} X(e^{\sigma+j\omega}) e^{j\omega n} d\omega
\]
Multiplying both sides by $e^{\sigma n}$, we get:
\[ 
x[n]=\frac{1}{2\pi}\int_0^{2\pi} X(e^{\sigma+j\omega}) e^{(\sigma+j\omega) n} d\omega 
=\frac{1}{2\pi}\int_0^{2\pi} X(z) z^n d\omega 
\]
To represent the inverse transform in terms of $z$ (instead of $\omega$), we note
\[ dz=d(e^{\sigma+j\omega})=e^\sigma j\;e^{j\omega} d\omega=jz\;d\omega,
 \;\;\;\;\;\mbox{i.e.,}\;\;\;\;\;\;d\omega=\frac{dz}{jz}	\]
and the inverse Z transform can be obtained as:
\[ x[n]={\cal Z}^{-1}[X(z)]=\frac{1}{j2\pi}\oint X(z) z^{n-1} dz	\]
Note that the integral with respect to $\omega$ from $0$ to $2\pi$ becomes
an integral with respect to $z=e^{\sigma+j\omega}$ in the complex z-plane,
along a circle with a fixed radius $e^\sigma$ and a varying angle $\omega$ 
from $0$ to $2\pi$. Now we have the z-transform pair:
\[	X(z)={\cal Z}[x[n]]=\sum_{n=-\infty}^\infty x[n]z^{-n}	\]
\[	x[n]={\cal Z}^{-1}[X(z)]=\frac{1}{2\pi j}\oint X(z)z^{n-1} dz	\]
The forward and inverse z-transform pair can also be represented as
\[	x[n] \stackrel{ {\cal Z} }{\longleftrightarrow} X(z)	\]
In particular, if we let $\sigma=0$, i.e., $z=e^{j\omega}$, then the Z
transform becomes the discrete-time Fourier transform:
\[ X(z)\bigg|_{z=e^{j\omega}}=\sum_{n=-\infty}^\infty x[n] e^{-j\omega n}=X(e^j\omega) \]
This is the reason why sometimes the discrete Fourier spectrum is expressed 
as a function of $e^{j\omega}$.

\htmladdimg{../ZFourier.gif}

Different from the discrete-time Fourier transform which converts a 1-D signal
$x[n]$ in time domain to a 1-D complex spectrum $X(e^{j\omega})$ in frequency
domain, the Z transform $X(s)$ converts the 1D signal $x[n]$ to a complex 
function defined over a 2-D {\em complex plane}, called z-plane, represented
in polar form by radius $|z|=|e^{\sigma+j\omega}|=e^\sigma$ and angle 
$\angle z=\angle(e^{\sigma+j\omega})=\omega$. 

In particular, when this 2D function $X(z)=X(e^{\sigma+j\omega})$ is evaluated 
along the unit circle $|z|=e^0=1$ corresponding to $\sigma=0$, it becomes a 
1D periodic function $X(e^{j\omega})$, the discrete Fourier transform of $x[n]$. 
Graphically, the periodic spectrum of the signal can be found as the cross 
section of the 2D function $X(z)=X(e^{\sigma+j\omega})$ along the unit circle
$|z|=e^0$.

{\bf Transfer Function of LTI system}

The output $y[n]$ of a discrete LTI system with input $x[n]$ can be found 
by convolution
\[ y[n]={\cal O}[x[n]]=h[n]*x[n]=\sum_{k=-\infty}^\infty h[k]x[n-k] 	\]
where $h[n]$ is the {\em impulse response function} of the system. In 
particular, if the input is a complex exponential
\[	x[n]=z^n=(e^s)^n=e^{sn}	\]
then the output $y[n]$ can be found to be:
\[	y[n]={\cal O}[z^n]=\sum_{k=-\infty}^\infty h[k]z^{n-k}
	=z^n \sum_{k=-\infty}^\infty h[k]z^{-k}=H(z) z^n	\]
This is the eigenequation with the complex exponential $x[n]=z^n=e^{sn}$ being
the eigenfunction of {\em any} discrete LTI system, corresponding to its 
eigenvalue defined as:
\[	H(z)\stackrel{\triangle}{=}\sum_{k=-\infty}^\infty h[k] z^{-k} \]
which is the {\em z-transform} of its impulse response $h[n]$, called the
{\em transfer function} of the LTI system. In particular, when $\sigma=0$,
i.e., $z=e^s=e^{j\omega}$, the transfer function $H(z)$ becomes the 
{\em frequency response function}, the Fourier transform of the impulse 
response:
\[ H(e^{j\omega})\stackrel{\triangle}{=}\sum_{n=-\infty}^\infty h[n] e^{-j\omega n} \]

\newpage
\section*{Conformal Mapping between S-Plane to Z-Plane}

The s-plane and the z-plane are related by a {\em conformal mapping} specified
by the analytic complex function 
\[	z=e^s=e^{\sigma+j\omega}=e^\sigma e^{j\omega}=r e^{j\omega}	\]
where 
\[ \left\{ \begin{array}{l} Re[s]=\sigma \\ Im[s]=j\omega \end{array} \right.
	\;\;\;\;\;\mbox{and}\;\;\;\;
	\left\{ \begin{array}{l} |z|=r=e^\sigma \\ \angle{z}=\omega 
	\end{array} \right.
\]
The mapping is continuous, i.e., neighboring points in s-plane are mapped
to neighboring points in z-plane and vice versa. Consider the mapping of these 
specific features: 
\begin{itemize}
\item The origin $s=0$ of s-plane is mapped to $z=e^0=1$ on the real axis in
	z-plane.
\item Each vertical line $Re[s]=\sigma_0$ in s-plane is mapped to a circle 
	$|z|=e^{\sigma_0}$ centered about the origin in z-plane. In particular,
	\begin{itemize}
	\item Leftmost vertical line $Re[s]=\sigma=-\infty$ is mapped as the 
		origin $|z|=e^{-\infty}=0$
	\item Imaginary axis $Re[s]=0$ is mapped as the unit circle 
		$|z|=e^0=1$
	\item Rightmost vertical line $Re[s]=\sigma=\infty$ is mapped as a 
		circle of infinite radius $|z|=e^{\infty}=\infty$.
	\end{itemize}
\item Each horizontal line $Im[s]=j\omega_0$ in s-plane is mapped to 
	$\angle{z}=\omega_0$, a ray from the origin in z-plane of angle 
	$\omega_0$ with respect to the positive horizontal direction. 
\item A right angle formed by a pair vertical and horizontal lines in s-plane
	is conserved by the mapping, as the corresponding circle and ray in z-plane 
	also form a right angle. (In fact any angle is conserved, an important
	property of the conformal mapping.)
\end{itemize}
The infinite range $-\infty < \omega < \infty$ for frequency $\omega$ along a 
vertical line $Re[s]=\sigma_0$ in s-plane is mapped {\em repeatedly} to a finite 
range $0 \le \omega < 2\pi$ around a circle $|z|=e^{\sigma_0}$ in z-plane, 
corresponding to the conversion of a continuous signal $x(t)$ with non-periodic 
spectrum $X(j\omega)$ for $-\infty < \omega < \infty$ to a discrete signal $x[n]$ 
with periodic spectrum $X(e^{j\omega})$ for $0 \le \omega < 2\pi$.

\htmladdimg{../conformal.gif}

\newpage
\section*{Region of Convergence and Examples}

Whether the z-transform $X(z)$ of a signal $x[n]$ exists depends on the 
complex variable $z=e^s$ as well as the signal itself. $X(z)$ exists if 
and only if the argument $z$ is inside the {\em region of convergence 
(ROC)} in the z-plane, which is composed of all $z$ values for the
summation of the Z-transform to converge. The ROC of the Z-transform is 
determined by $|z|=|e^s|=e^{\sigma}$ (a circle), the magnitude of variable
$z$, while the ROC for the Laplace transform is determined by $\sigma=Re[s]$,
(a vertical line), the real part of $s$. This formula is always needed in 
the examples:
\[	\sum_{n=0}^\infty x^n=\frac{1}{1-x},\;\;\;\;\mbox{for $|x|<1$} \]

{\bf Example 1:} The Z transform of a right sided signal $x[n]=a^n u[n]$ is
\[	X(z)=\sum_{n=-\infty}^\infty a^n u[n] z^{-n}
	=\sum_{n=0}^\infty (az^{-1})^n=\frac{1}{1-az^{-1}}=\frac{z}{z-a} \]
For this summation to converge, i.e., for $X(z)$ to exist, it is necessary
to have $| az^{-1} |<1$, i.e., the ROC is $|z| > |a|$. As a special case when
$a=1$, $x[n]=u[n]$ and we have
\[	{\cal Z}[u[n]]=\frac{1}{1-z^{-1}}, \;\;\;\;|z|>1	\]

{\bf Example 2:} The Z-transform of a left sided signal $x[n]=-a^nu[-n-1]$ is:
\begin{eqnarray}
X(z) &=& -\sum_{n=-\infty}^\infty a^nu[-n-1]z^{-n}=-\sum_{n=-\infty}^{-1} (az^{-1})^n
	\nonumber \\
 &=& 1-\sum_{n=0}^\infty (a^{-1}z)^n=1-\frac{1}{1-a^{-1}z}=\frac{z}{z-a}=\frac{1}{1-az^{-1}}
	\nonumber
\end{eqnarray}
For the summation above to converge, it is required that $|a^{-1}z|<1$, i.e.,
the ROC is $|z|<|a|$. Comparing the two examples above we see that two different
signals can have identical z-transform, but with different ROCs. 

{\bf Example 3: } Find the inverse of the given z-transform $X(z)=4z^2+2+3z^{-1}$.
Comparing this with the definition of z-transform:
\[
X(z)=\sum_{n=-\infty}^\infty x[n]z^{-n}=x[-2]z^2+x[-1]z^1+x[0]+x[1]z^{-1}+x[2]z^{-2}
\]
we get
\[	x[n]=4\delta[n+2]+2\delta[n]+3\delta[n-1]	\]
In general, we can use the time shifting property
\[	{\cal Z}[\delta[n+n_0]]=z^{n_0}	\]
to inverse transform the $X(z)$ given above to $x[n]$ directly.

{\bf Example 4: } Sometimes the inverse transform of a given $X(z)$ can be
obtained by long division.
\[ X(z)=\frac{1}{1-az^{-1}} \]
By a long division, we get
\[ 1\div (1-az^{-1})=1+az^{-1}+a^2z^{-2}+\cdots \]
which converges if the ROC is $|z|>|a|$, i.e., $|az^{-1}|<1$ and we get
\[ x[n]=a^n u[n] \].
Alternatively, the long division can also be carried out as:
\[ 1\div (-az^{-1}+1)=-a^{-1}z-a^{-2}z^2-\cdots \]
which converges if the ROC is $|z|<|a|$, i.e., $|a^{-1}z|<1$ and we get
\[ x[n]=-a^n u[-1-n] \]

\newpage
\section*{Zeros and Poles of Z-Transform}

All z-transforms in the above examples are rational, i.e., they can 
be written as a ratio of polynomials of variable $z$ in the general form
\[	
X(z)=\frac{N(z)}{D(z)}=\frac{\sum_{k=0}^M b_k z^k }{\sum_{k=0}^N a_k z^k}
=\frac{b_M}{a_N}\frac{\prod_{k=1}^M (z-z_{z_k})}{\prod_{k=1}^N (z-z_{p_k})} 
\]
where $N(z)$ is the numerator polynomial of order $M$ with roots $z_{z_k}, (k=1,2, 
\cdots, M)$, and $D(z)$ is the denominator polynomial of order $N$ with roots 
$z_{p_k}, (k=1,2, \cdots, N)$. In general, we assume the order of the numerator
polynomial is lower than that of the denominator polynomial, i.e.,  $M < N$. If 
this is not the case, we can always expand $X(z)$ into multiple terms so that $M<N$
is true for each of terms.

The {\em zeros} and {\em poles} of a rational $X(z)=N(z)/D(z)$ are defined as:
\begin{itemize}
\item {\bf Zero}: 
  Each of the roots of the numerator polynomial $z_z$ for which 
  $X(z)\bigg|_{z=z_z}=X(z_z)=0$ is a {\em zero} of $X(z)$.

  If the order of $D(z)$ exceeds that of $N(z)$ (i.e., $N>M$), then 
  $X(\infty)=0$, i.e., there is a zero at infinity:
  \[	\frac{b_1z+b_0}{a_2z^2+a_1z+a_0} \bigg|_{z \rightarrow \infty} =0	\]

\item {\bf Pole}:
  Each of the roots of the denominator polynomial $z_p$ for which 
  $X(z)\bigg|_{z=z_p}=X(z_p)=\infty$ is a {\em pole} of $X(z)$.

  If the order of $N(z)$ exceeds that of $D(z)$ (i.e., $M>N$), then 
  $X(\infty)=\infty$, i.e, there is a pole at infinity: 
  \[	\frac{b_2z^2+b_1z+b_0}{a_1z+a_0} \bigg|_{z \rightarrow \infty} \rightarrow \infty \]

\end{itemize}
Most essential behavior properties of an LTI system can be obtained graphically
from the ROC and the zeros and poles of its transfer function $H(z)$ on the z-plane.

\newpage
\section*{Properties of ROC}	

Whether the z-transform $X(z)$ of a function $x[n]$ exists depends on whether
or not the transform summation converges 
\[ X(z)=\sum_{n=-\infty}^\infty x[n]z^{-n}
	=\sum_{n=-\infty}^\infty x[n] (e^{\sigma})^{-n} e^{-j\omega n}
	=\sum_{n=-\infty}^\infty x[n] |z|^{-n} e^{-j\omega n} < \infty \]
which in turn depends on the duration and magnitude of $x[n]$ as well as the 
magnitude $|z|=r=e^\sigma$ (the phase of $z$ $\angle{z}=\omega$ determines the 
frequency of a sinusoid which is bounded and has no effect on the convergence 
of the integral). 

{\bf Right sided signals: } $x[n]=x[n]u[n-n_0]$ may have infinite duration for 
$n>0$, and a $z$ value with $|z|=e^{\sigma}>1$ tends to attenuate $x[n]|z|^{-n}$ as 
$n \rightarrow \infty$. 

{\bf Left sided signals: } $x[n]u[n_0-n]$ may have infinite duration for $n<0$, 
and a $z$ value with $|z|=e^{\sigma}<1$ tends to attenuate $x[n]|z|^{-n}$ as 
$n \rightarrow -\infty$.  

Based on these observations, we can get the following properties for the ROC:

\begin{itemize}

\item If $x[n]$ is of finite duration, then the ROC is the entire z-plane (the
	z-transform summation converges, i.e., $X(z)$ exists, for any $z$) except
	possibly $z=0$ and/or $z=\infty$.

\item The ROC of $X(z)$ consists of a ring centered about the origin in the z-plane.
	The inner boundary can extend inward to the origin in some cases, and the 
	outer can extend to infinity in other cases.

\item If $x[n]$ is right sided and the circle $|z|=r_0$ is in the ROC, then any 
	finite $z$ for which $|z|>r_0$ is also in the ROC. 

\item If $x[n]$ is left sided and the circle $|z|=r_0$ is in the ROC, then any
	$z$ for which $0<|z|<r_0$ is also in the ROC.

\item If $x[n]$ is two-sided, then the ROC is the intersection of the two one-sided 
	ROCs corresponding to the two one-sided parts of $x[n]$. This intersection
	can be either a ring or an empty set.

\item If $X(z)$ is rational, then its ROC does not contain any poles (by definition 
	$X(z)\bigg|_{z=z_p}=\infty$ dose not exist). The ROC is bounded by the poles or 
	extends to infinity.

\item If $X(z)$ is a rational z-transform of a right sided function $x[n]$, then
	the ROC is the region outside the out-most pole. If $x[n]=0$ for $n<0$
	(causal), then the ROC includes $z=\infty$. 

\item If $X(z)$ is a rational z-transform of a left sided function $x[n]$, then 
	the ROC is inside the innermost pole. If $x[n]=0$ for $n \ge 0$ 
	(anti-causal), then the ROC includes $z=0$. 

\item Fourier transform $X(e^{j\omega})$ of discrete signal $x[n]$ exists if the
	ROC of the corresponding z-transform $X(z)$ contains the unit circle
	$|z|=1$ or $z=e^{j\omega}$.

\end{itemize}

\newpage
{\bf Example 1: } 
\[	X(z)=\sum_{n=-3}^5 x[n]z^{-n}	\]
	When $z=0$, $z^{-n}=\infty$ for $n>0$, when $z=\infty$, $z^{-n}=\infty$
	for $n<0$. Therefore neither $z=0$ nor $|z|=\infty$ are included in the ROC.


{\bf Example 2:} 
\[ x[n]=a^{|n|}=a^n u[n]+a^{-n} u[-n-1] \]
The Z-transform is linear, and $X(z)$ is the sum of the transforms for the two terms:
\[ {\cal Z}[a^n u[n]]=\frac{1}{1-az^{-1}},\;\;\;\;\;(|z|>|a|),\;\;\;\;\;\;\;\;\;
   {\cal Z}[a^{-n}u[-n-1]=\frac{-1}{1-a^{-1}z^{-1}},\;\;\;\;\;(|z|<1/|a|) \]
If $|a|<1$, i.e., $x[n]$ decays when $|n|\rightarrow\infty$, the intersection of
the two ROCs is $|a|<|z|<1/|a|$, and we have:
\[ {\cal Z}[x[n]]=\frac{1}{1-az^{-1}}-\frac{1}{1-a^{-1}z^{-1}}
=\frac{a^2-1}{a}\frac{z}{(z-a)(z-1/a)} \]
However, if $|a|>1$, i.e., $x[n]$ grows without a bound when $|n|\rightarrow\infty$,
the intersection of the two ROCs is a empty set, the Z-transform does not exist.

{\bf Example 3: } Given the following z-transform, find the corresponding signal:
\[ X(z)=\frac{1}{(1-\frac{1}{3}z^{-1})(1-2z^{-1})}
	=-\frac{1/5}{1-\frac{1}{3}z^{-1}}+\frac{6/5}{1-2z^{-1}} \]
The two poles are $z_{p_1}=1/3$ and $z_{p_2}=2$, respectively. The $X(z)$ has 
three possible ROCs associated with three different time signals $x[n]$:
\begin{itemize}
\item The region outside the out-most pole $z_{p_2}=2$, with the corresponding 
	right sided time function
\[ x[n]=-\frac{1}{5}(\frac{1}{3})^n u[n]+\frac{6}{5}2^n u[n]	\]
\item The region inside the innermost pole $z_{p_1}=1/3$, with the corresponding 
	left sided time function
\[ x[n]=\frac{1}{5}(\frac{1}{3})^n u[-n-1]-\frac{6}{5}2^n u[-n-1]	\]
\item The ring between the two poles $1/3 < |z| < 2$, with the corresponding two 
	sided time function
\[ x[n]=-\frac{1}{5}(\frac{1}{3})^n u[n]-\frac{6}{5}2^n u[-n-1]	\]
\end{itemize}
In particular, note that only the last ROC includes the circle $|z|=1$ and the 
corresponding time function $x[n]$ has a discrete Fourier transform. Fourier 
transform of the other two functions do not exist.

\newpage
\section*{Properties of Z-Transform}

The z-transform has a set of properties in parallel with that of the Fourier 
transform (and Laplace transform). The difference is that we need to pay 
special attention to the ROCs. In the following, we always assume
\[	{\cal Z}[x[n]]=X(z)\;\;\;\;ROC=R_x	\]
and
\[	{\cal Z}[y[n]]=Y(z)\;\;\;\;ROC=R_y	\]

\begin{itemize}
\item {\bf Linearity}
\[
{\cal Z}[a x[n]+b y[n]]=aX(z)+bY(z), \;\;\;\;ROC \supseteq (R_x \cap R_y) \]
While it is obvious that the ROC of the linear combination of $x[n]$ and 
$y[n]$ should be the intersection of the their individual ROCs $R_x \cap R_y$
in which both $X(z)$ and $Y(z)$ exist, note that in some cases the ROC of the
linear combination could be larger than $R_x \cap R_y$. For example, for both 
$x[n]=a^n u[n]$ and $y[n]=a^n u[n-1]$, the ROC is $|z|>|a|$, but the ROC of 
their difference $a^n u[n]-a^n u[n-1]=\delta[n]$ is the entire z-plane.

\item {\bf Time Shifting}
\[	{\cal Z}[x[n-n_0]]=z^{-n_0} X(z),\;\;\;\;ROC=R_x 	\]
{\bf Proof: }
\[	{\cal Z}[x[n-n_0]]=\sum_{n=-\infty}^\infty x[n-n_0]z^{-n}	\]
Define $m=n-n_0$, we have $n=m+n_0$ and
\[	\sum_{m=-\infty}^\infty x[m]z^{-m}z^{-n_0}=z^{-n_0} X(z)	\]
The new ROC is the same as the old one except the possible addition/deletion 
of the origin or infinity as the shift may change the duration of the signal.

\item {\bf Time Expansion (Scaling)}

\[	{\cal Z}[x[n/k]]=X(z^k),\;\;\;\;ROC=R_x^{1/k}	\]
The discrete signal $x[n]$ cannot be continuously scaled in time as $n$ has 
to be an integer (for a non-integer $n$ $x[n]$ is zero). Therefore $x[n/k]$ is
defined as
\[	x[n/k]\stackrel{\triangle}{=}\left\{ \begin{array}{ll}
	x[n/k] & \mbox{if $n$ is a multiple of $k$} \\ 0 & \mbox{else}
	\end{array} \right.
\]
{\bf Example: } If $x[n]$ is ramp

\begin{tabular}{c|cccccc} \hline
 $n$ & 1 & 2 & 3 & 4 & 5 & 6 \\ \hline $x[n]$ & 1 & 2 & 3 & 4 & 5 & 6 \\ \hline 
\end{tabular}

then the expanded version $x[n/2]$ is 

\begin{tabular}{c|cccccc} \hline
 $n$ & 1 & 2 & 3 & 4 & 5 & 6 \\ \hline
 $n/2$ & 0.5 & 1 & 1.5 & 2 & 2.5 & 3 \\ \hline
 $m$ &	     & 1 &     & 2 &     & 3 \\ \hline
 $x[n/2]$ & 0 & 1 & 0 & 2 & 0 & 3 \\ \hline 
\end{tabular}

where $m$ is the integer part of $n/k$. 

{\bf Proof: } The z-transform of such an expanded signal is
\[	{\cal Z}[x[n/k]]=\sum_{n=-\infty}^\infty x[n/k]z^{-n}	
	=\sum_{m=-\infty}^\infty x[m]z^{-km}=X(z^k)	\]
Note that the change of the summation index from $n$ to $m$ has no effect as the 
terms skipped are all zeros.

\item {\bf Convolution}
\[	{\cal Z}[x[n]*y[n]]=X(z) Y(z), \;\;\;\;ROC \supseteq (R_x \cap R_y) \]
The ROC of the convolution could be larger than the intersection of $R_x$ and
$R_y$, due to the possible pole-zero cancellation caused by the convolution.

\item {\bf Time Difference}
\[	{\cal Z}[x[n]-x[n-1]]=(1-z^{-1})X(z), \;\;\;\;\;ROC=R_x  	\]
{\bf Proof: } 
\[	
{\cal Z}[x[n]-x[n-1]]=X(z)-z^{-1}X(z)=(1-z^{-1})X(z)=\frac{z-1}{z}X(z)
\]
Note that due to the additional zero $z=1$ and pole $z=0$, the resulting ROC 
is the same as $R_x$ except the possible deletion of $z=0$ caused by the added
pole and/or addition of $z=1$ caused by the added zero which may cancel an 
existing pole.

\item {\bf Time Accumulation}
\[	{\cal Z}[\sum_{k=-\infty}^nx[k]]=\frac{1}{1-z^{-1}}X(z),
	\;\;\;\;ROC \supseteq [R_x \cap (|z|>1)]	\]	
{\bf Proof: } The accumulation of $x[n]$ can be written as its convolution with $u[n]$:
\[
u[n]*x[n]=\sum_{k=-\infty}^\infty u[n-k]x[k]=\sum_{k=-\infty}^n x[k]
\]
Applying the convolution property, we get
\[
{\cal Z}[\sum_{k=-\infty}^nx[k]]={\cal Z}[u[n]*x[n]]=\frac{1}{1-z^{-1}}X(z)
\]
as ${\cal Z}[u[n]]=1/(1-z^{-1})$.

\item {\bf Time Reversal}
\[	{\cal Z}[x[-n]]=X(1/z)\;\;\;\;ROC=1/R_x	\]
{\bf Proof: }
\[	{\cal Z}[x[-n]]=\sum_{n=-\infty}^\infty x[-n]z^{-n}	
	=\sum_{m=-\infty}^\infty x[m](\frac{1}{z})^{-m}=X(1/z)
\]
where $m=-n$.

\item {\bf Scaling in Z-domain}

\[	{\cal Z}[a^n x[n]]=X\left(\frac{z}{a}\right),\;\;\;\;ROC=|a|R_x	\]
{\bf Proof: }
\[	{\cal Z}[a^n x[n]]=\sum_{n=-\infty}^\infty x[n]\left(\frac{z}{a} \right)^{-n}
	=X\left(\frac{z}{a}\right)
\]
In particular, if $a=e^{j\omega_0}$, the above becomes
\[	{\cal Z}[e^{jn\omega_0} x[n]]=X(e^{-j\omega_0}z)\;\;\;\;ROC=R_x	\]
The multiplication by $e^{-j\omega_0}$ to $z$ corresponds to a rotation by 
angle $\omega_0$ in the z-plane, i.e., a frequency shift by $\omega_0$.
The rotation is either clockwise ($\omega_0>0$) or counter clockwise 
($\omega_0<0$) corresponding to, respectively, either a left-shift or a
right shift in frequency domain. The property is essentially the same as 
the frequency shifting property of discrete Fourier transform. 

\item {\bf Conjugation}
\[	{\cal Z}[x^*[n]]=X^*(z^*), \;\;\;\;\;ROC=R_x	\]
{\bf Proof: } Complex conjugate of the z-transform of $x[n]$ is
\[	X^*(z)=[\sum_{n=-\infty}^\infty x[n]z^{-n}]^*
	=\sum_{n=-\infty}^\infty x^*[n](z^*)^{-n}	\]
Replacing $z$ by $z^*$, we get the desired result.

\item {\bf Differentiation in z-Domain}
\[	{\cal Z}[n x[n]]=-z\;\frac{d}{dz}X(z), \;\;\;\; ROC=R_x	\]
{\bf Proof: } 
\[ \frac{d}{dz}X(z)=\sum_{n=-\infty}^\infty x[n] \frac{d}{dz} (z^{-n})
=\sum_{n=-\infty}^\infty (-n) x[n] z^{-n-1}
=\frac{-1}{z}\sum_{n=-\infty}^\infty n x[n] z^{-n} \]
i.e.,
\[	{\cal Z}[n x[n]]=-z\frac{d}{dz}X(z)	\]

{\bf Example: } Taking derivative with respect to $z$ of the right side of
\[	{\cal Z}[a^nu[n]]=\frac{1}{1-az^{-1}}\;\;\;\;|z|>|a|	\]
we get
\[	\frac{d}{dz}\left[ \frac{1}{1-az^{-1}}\right]=\frac{-az^{-2}}{(1-az^{-1})^2}	\]
Due to the  property of differentiation in z-domain, we have
\[	{\cal Z}[n a^n u[n]]=\frac{az^{-1}}{(1-az^{-1})^2}\;\;\;\;|z|>|a|	\]
Note that for a different ROC $|z|<|a|$, we have
\[	{\cal Z}[-n a^n u[-n-1]]=\frac{az^{-1}}{(1-az^{-1})^2}\;\;\;\;|z|<|a|	\]

\end{itemize}

%} {\large
\newpage
\section*{Z-Transform of Typical Signals}
\begin{itemize}
\item $\delta[n]$, $\delta[n-m]$
\[	{\cal Z}[\delta[n]]=\sum_{n=-\infty}^\infty \delta[n]z^{-n}=1
	\;\;\;\mbox{for all $z$} \]
Due to the time shifting property, we also have
\[	{\cal Z}[\delta[n-m]]=z^{-m}\;\;\;\mbox{for all $z$} \]

\item $u[n]$, $a^n u[n]$, $n a^n u[n]$

\[	{\cal Z}[u[n]]=\sum_{n=0}^\infty z^{-n}=\frac{1}{1-z^{-1}}\;\;\;|z|>1 \]
Due to the scaling in z-domain property, we have
\[	{\cal Z}[a^nu[n]]=\frac{1}{1-(z/a)^{-1}}=\frac{1}{1-az^{-1}}\;\;\;|z|>|a| \]
Applying the property of differentiation in z-Domain to the above, we have
\[	{\cal Z}[na^nu[n]]=-z\frac{d}{dz}[\frac{1}{1-az^{-1}}]
	=-z\frac{-az^{-2}}{(1-az^{-1})^2}=\frac{az^{-1}}{(1-az^{-1})^2}
	\;\;\;\;|z|>|a|	\]

\item $e^{\pm jn\omega_0}u[n]$, $cos[n\omega_0]u[n]$, $sin[n\omega_0]u[n]$

Applying the scaling in z-domain property to ${\cal Z}[u[n]]=1/(1-z^{-1})$, we have
\[	{\cal Z}[e^{jn\omega_0}u[n]]=\frac{1}{1-(e^{j\omega_0}z)^{-1}}
	=\frac{1}{1-e^{-j\omega_0} z^{-1}}	\;\;\;|z|>1	\]
and similarly, we have
\[	{\cal Z}[e^{-jn\omega_0}u[n]]=\frac{1}{1-e^{j\omega_0} z^{-1}}\;\;\;|z|>1 \]
Moreover, we have
\begin{eqnarray}
{\cal Z}[cos(n\omega_0)u[n]]&=&{\cal Z}[\frac{e^{jn\omega_0}+e^{-jn\omega_0}}{2}u[n]]
=\frac{1}{2}[\frac{1}{1-e^{j\omega_0} z^{-1}}+\frac{1}{1-e^{-j\omega_0} z^{-1}}]
	\nonumber \\
&=&\frac{2-(e^{j\omega_0}+e^{-j\omega_0})z^{-1}}{2[1-(e^{j\omega_0}+e^{-j\omega_0})z^{-1}+z^{-2}]}
	\nonumber \\
&=& \frac{1-cos\omega_0 z^{-1}}{1-2cos\omega_0 z^{-1}+z^{-2}}\;\;\;\;|z|>1
	\nonumber
\end{eqnarray}
Similarly we have
\[	{\cal Z}[sin(n\omega_0)u[n]]=
\frac{sin\omega_0 z^{-1}}{1-2cos\omega_0 z^{-1}+z^{-2}}	\;\;\;\;|z|>1 \]

\item $r^n cos[n\omega_0]u[n]$, $r^n sin[n\omega_0]u[n]$

Applying the z-domain scaling property to the above, we have
\[	{\cal Z}[r^n cos(n\omega_0)u[n]]=
\frac{1-r\;cos\omega_0 z^{-1}}{1-2r\;cos\omega_0 z^{-1}+r^2 z^{-2}} \;\;\;\;|z|>r	\]
and
\[	{\cal Z}[r^n sin(n\omega_0)u[n]]=
\frac{r\;sin\omega_0 z^{-1}}{1-2r\;cos\omega_0 z^{-1}+r^2 z^{-2}} \;\;\;\;|z|>r	\]

\end{itemize}


\newpage
\section*{Analysis of LTI Systems by Z-Transform}

Due to its convolution property, the z-transform is a powerful tool to analyze LTI
systems 
\[	y[n]=h[n]*x[n] \stackrel{{\cal Z}}{\longrightarrow} Y(z)=H(z)X(z)	\]
As discussed before, when the input is the eigenfunction of all LTI system, i.e., 
$x[n]=e^{sn}=z^n$, the operation on this input by the system can be found by 
multiplying the system's eigenvalue $H(z)$ to the input:
\[	y[n]={\cal O}[z^n]=h[n]*z^n=H(z) z^n	\]

\begin{itemize}
\item {\bf Causal LTI systems}

  An LTI system is {\em causal} if its output $y[n]$ depends only on the current 
  and past input $x[n]$ (but not the future). Assuming the system is initially at
  rest with zero output $y[n]\bigg|_{n<0}=0$, then its response $y[n]=h[n]$ to an
  impulse $x[n]=\delta[n]$ at $n=0$ is at rest for $n<0$, i.e., $h[n]=h[n]u[n]$.
  Its response to a general input $x(t)$ is:
  \[ y[n]=h[n]*x[n]=\sum_{m=-\infty}^\infty h[m] x[n-m]=\sum_{m=0}^\infty h[m] x[n-m] \]
  Due to the properties of the ROC, we know that 

  {\bf If an LTI system is causal (with a right sided impulse response function
    $h[n]=0$ for $n<0$), then the ROC of its transfer function $H(z)$ is the
    exterior of a circle including infinity. In particular, when $H(z)$ is rational,
    then the system is causal if and only if its ROC is the exterior of a circle 
    outside the out-most pole, and the order of numerator is no greater than the order
    of the denominator.}

  Note the requirement for the orders of the numerator and denominator guarantees
  the existence of $H(z)$ even when $z=\infty$.

\item {\bf Stable LTI systems}

  An LTI system is {\em stable} if its response to any bounded input is also 
  bounded for all $n$:
  \[  \mbox{if}\;\;|x[n]|<B_x\;\;\;\mbox{then}\;\;\;|y[n]|<\infty  \]
  As the output and input of an LTI is related by convolution, we have:
  \[  y[n]=h[n]*x[n]=\sum_{m=-\infty}^\infty h[m] x[n-m]<\infty \]
  and 
  \[ |y[n]| &=& |\sum_{m=-\infty}^\infty h[m] x[n-m]| 
  \le \sum_{m=-\infty}^\infty |h[m] |x[n-m]| 
  <B_x\;\sum_{m=-\infty}^\infty |h[m]| <\infty	\]
  which obviously requires:
  \[  \sum_{m=-\infty}^\infty |h[m]| <\infty	 \]
  In other words, if the impulse response function $h[m]$ of an LTI system is 
  absolutely integrable, then the system is stable. We can show that this 
  condition is also necessary, i.e., all stable LTI systems' impulse response
  functions are absolutely integrable. Now we have:
  
  {\bf An LTI system is stable if and only if its impulse response is absolutely
    summable, i.e., the frequency response function $H(e^{j\omega})$ exits, i.e. the
    ROC of its transfer function $H(z)$ includes the unit circle $|z|=1$.}
  \[ H(z)\bigg|_{z=e^{j\omega}}=H(e^{j\omega})={\cal F}[h[n]] \]

\item {\bf Causal and stable LTI systems}

  From the two properties above, we also see that

  {\bf A causal LTI system with a rational transfer function $H(z)$ is stable 
    if and only if all poles of $H(z)$ are inside the unit circle of the z-plane, 
    i.e., the magnitudes of all poles are smaller than 1.}
  \[ |Re[z_p]|<1\;\;\;\;\;\;\;\mbox{(for all $z_p$)} \]


\end{itemize}

{\bf Example: } The transfer function of an LTI is
\[	H(z)=\frac{1}{1-az^{-1}}	\]
As shown before, without specifying the ROC, this $H(z)$ could be the z-transform
of one of the two possible time signals $h[n]$.
\begin{itemize}
\item If ROC is $|z|>|a|$, the system $h[n]=a^n u[n]$ is causal.
	\begin{itemize}
	\item If $|a|<1$, i.e., unit circle can be included in ROC, the 
		system is stable;
	\item If $|a|>1$, i.e., unit circle cannot be included in ROC, the 
		system is unstable;
	\end{itemize}
\item If ROC is $|z|<|a|$, the system $h[n]=-a^n u[-n-1]$ is anti-causal.
	\begin{itemize}
	\item If $|a|<1$, i.e., unit circle cannot be included in ROC, the 
		system is unstable;
	\item If $|a|>1$, i.e., unit circle can be included in ROC, the 
		system is stable;
	\end{itemize}
\end{itemize}

\vskip 0.2in
\begin{tabular}{c||c|c} \hline 
		& $|a|<1$			& $|a|>1$		\\ \hline \hline
  $|z|>|a|$	& $e^{j\omega}$ inside ROC & $e^{j\omega}$ outside ROC, \\
  $h[n]=e^a^nu[n]$& causal, stable & causal, unstable  \\ \hline
  $|z|<|a|$	& $e^{j\omega}$ outside ROC & $e^{j\omega}$ inside ROC, \\
  $h[n]=-a^nu[-n-1]$ 	& anti-causal, unstable & anti-causal, stable  \\ \hline
\end{tabular}

\section*{LTI Systems Characterized by LCCDEs}

The first order difference is defined as 
\[ dx[n]=x[n+1]-x[n],\;\;\;\;\;\mbox{or}\;\;\;\;\;\;dx[n]=x[n]-x[n-1] \]
The second order difference is defined as 
\[ d^2 x[n]=dx[n]-dx[n-1]=x[n]-x[n-1]-x[n-1]+x[n-2]=x[n]-2x[n-1]+x[n-2] \]
In general, the kth order difference will need to involve $x[n-k]$. 

Similar to an LTI continuous system which can be described by an LCCDE 
(differential equation), a discrete LTI system can be described by an LCCDE 
(difference equation):
\[	\sum_{k=0}^N a_k y[n-k]=\sum_{k=0}^M b_k x[n-k]	\]
where $x[n]$ and $y[n]$ are the discrete input and output, respectively.
After taking the z-transform on both sides of the LCCDE, we get an algebraic 
equation in the $z$ domain:
\[ Y(z)[\sum_{k=0}^N a_k z^{-k}]=X(z)[\sum_{k=0}^M b_k z^{-k}]	\]
and its transfer function is rational:
\[
H(z)=\frac{Y(z)}{X(z)}=\frac{\sum_{k=0}^M b_k z^{-k}}{\sum_{k=0}^N a_k z^{-k}}
=\frac{b_M}{a_N} \; \frac{\prod_{k=1}^M (z-z_{z_k})}{\prod_{k=1}^N (z-z_{z_k})}
=K \; \frac{N(z)}{D(z)}	\]
where $K=b_M/a_N$ is a coefficient and $z_{z_k}, (k=1,2, \cdots, M)$ are the 
roots of the numerator polynomial and $z_{p_k}, (k=1,2, \cdots, N)$ are the 
roots of the denominator polynomial. Note that just as the LCCDE alone does 
not completely specify the relationship between $x[n]$ and $y[n]$ (additional 
information such as the initial conditions is needed), the transfer function 
$H(z)$ does not completely specify the system. For example, the same $H(z)$ 
with different ROCs will represent different systems (e.g., causal or anti-causal). 

{\bf Example:} The input and output of an LTI system are related by
\[	y[n]-\frac{1}{2}y[n-1]=x[n]+\frac{1}{3}x[n-1]	\]
Note that without further information such as the initial condition, this
equation does not uniquely specify $y[n]$ when $x[n]$ is given. Taking 
z-transform of this equation and using the time shifting property, we get
\[	Y(z)-\frac{1}{2}z^{-1}Y(z)=X(z)+\frac{1}{3}z^{-1}X(z)	\]
and the transfer function can be obtained
\[
H(z)=\frac{Y(z)}{X(z)}=\frac{1+\frac{1}{3}z^{-1}}{1-\frac{1}{2}z^{-1}}
=\frac{1}{1-\frac{1}{2}z^{-1}}(1+\frac{1}{3}z^{-1})	\]
Note that the causality and stability of the system is not provided by this
equation, unless the ROC of this $H(z)$ is specified. Consider these two 
possible ROCs:
\begin{itemize}
\item If ROC is $|z|>1/2$, it is outside the pole $z_p=1/2$ and includes 
	the unit circle. The system is causal and stable:
\[	h[n]=\left(\frac{1}{2}\right)^n u[n]+\frac{1}{3}\left(\frac{1}{2}\right)^{n-1}u[n-1]	\]
\item If ROC is $z|<1/2$, it is inside the pole $z_p=1/2$ and does not include
	the unit circle. The system is anti-causal and unstable:
\[	h[n]=-\left(\frac{1}{2}\right)^n u[-n-1]-\frac{1}{3}\left(\frac{1}{2}\right)^{n-1}u[-n]	\]
\end{itemize}

\newpage
\section*{Evaluation of Fourier Transform from Pole-Zero Plot}

Given the pole-zero plot of the transfer function $H(s)$, we can qualitatively 
learn the system's behavior as the frequency $\omega$ changes.

\subsection*{First order system}

The first order discrete system is described by
\[	y[n]-ay[n-1]=x[n] 	\]
The impulse response $h[n]$ can be found by solving the following
\[	h[n]-ah[n-1]=\delta[n] 	\]
to be
\[	h[n]=a^n u[n]	\]
Alternatively, we can take z-transform of the DE and get
\[	Y(z)-az^{-1}Y(z)=(1-az^{-1})Y(z)=X(z)	\]
and the transfer function of the system (assumed causal)
\[	H(z)=\frac{Y(z)}{X(z)}=\frac{1}{1-az^{-1}} \;\;\;\;\;\;|z|>|a|	\]
$H(z)$ has a zero at $z=0$ and a pole $z_p=a$ and its ROC is the region 
$|z|>|a|$ outside the pole. If $|a|<1$, then the unit circle $|z|=r=1$ can be
included in the ROC, the Fourier transform exists and the system is stable. The 
impulse response 
(unit sample response) of the system is
\[	h[n]={\cal Z}^{-1}[H(z)]=a^nu[n]	\]
Although $h[n]$ seems to have a form different from the typical impulse response
in continuous case $h(t)=e^{-t/\tau}u(t)$, they are essentially the same as $h(t)$
can be rewritten as
\[	h(t)=e^{-t/\tau}u(t)=(e^{-1/\tau})^t u(t)=a^t u(t)	\]
where 
\[	a \stackrel{\triangle}{=}e^{-1/\tau}	\]
Letting $z=e^{j\omega}$ in $H(z)$, we get the frequency response function of the
system
\[	H(e^{j\omega})=\frac{1}{1-ae^{-j\omega}}
	=\frac{e^{j\omega}}{e^{j\omega}-a}=\frac{e^{j\omega}-0}{e^{j\omega}-a}
	=\frac{u}{v}		\]
where $u$ and $v$ are two vectors in z-plane defined as
\[	u\stackrel{\triangle}{=}e^{j\omega}-0=e^{j\omega},\;\;\;\;
	v\stackrel{\triangle}{=}e^{j\omega}-a	\]
For any frequency $-\pi \le\omega\le\pi$ represented by a point $z=e^{j\omega}$
on the unit circle, the magnitude and phase angle of the frequency response
function can be represented in the z-plane as
\[	|H(e^{j\omega})|=\frac{|u|}{|v|}	\]
and 
\[	\angle H(e^{j\omega})=\angle u-\angle v	\]
which can be evaluated graphically in the z-plane as the frequency $\omega$ changes
in the range $-\pi \le \omega \le\pi$. If we assume $0<a< 1$, then when $\omega=0$,
the denominator reaches its minimum of $1-a$, and $|H(e^{j\omega})|$ is maximized
to be $1/(1-a)$; and when $\omega=\pi$, the denominator reaches its maximum of 
$|(-1-a)|=1+a$, and $|H(e^{j\omega})|$ is minimized to be $1/(1+a)$. 
The phase angle of $H(e^{j\omega})=\angle u-\angle v$ is zero when $\omega=0$ or
$\omega=\pi$, and is negative for $0<\omega<\pi$ and positive for $-\pi<\omega<0$.

\newpage
\subsection*{Second order system}

A second order discrete system is described by
\[	y[n]+a_1y[n-1]+a_2y[n-2]=x[n]	\]
The z-transform of this DE is
\[	Y(z)+a_1z^{-1}Y(z)+a_2z^{-2}Y(z)=(1+a_1z^{-1}+a_2z^{-2})Y(z)=X(z)	\]
and the transfer function is
\[	H(z)=\frac{Y(z)}{X(z)}=\frac{1}{1+a_1z^{-1}+a_2z^{-2}}
	=\frac{z^2}{z^2+a_1z+a_2}=\frac{z^2}{(z-z_{p_1})(z-z_{p_2})}
\]
which has two repeated zeros $z=0$ and a pair of complex conjugate poles
\[	z_{p_1,p_2}=-\frac{a_1}{2}\pm\frac{1}{2}\sqrt{a_1^2-4a_2}
		=-\frac{a_1}{2}\pm\frac{j}{2}\sqrt{4a_2-a_1^2}	\]
For convenience, we define
\[	a_1\stackrel{\triangle}{=}-2r\;cos\theta, \;\;\;\;\mbox{and}
\;\;\;\; a_2\stackrel{\triangle}{=}r^2\;\;\;\;(0\ge \theta \ge \pi) \]
and the transfer function becomes
\[	H(z)=\frac{z^2}{z^2-2r\;cos\theta z+r^2}	\]
and the poles become
\[
z_{p_1,p_2}=r(cos\theta \pm j\;sin\theta)=r\;e^{\pm j\theta}	\]
For the system to be causal and stable, we have to have $|z_p|=r<1$. 
When $0<\theta <\pi$, $H(z)$ can be expanded 
\begin{eqnarray}
 H(z) &=& \frac{z^2}{(z-z_{p_1})(z-z_{p_2})}=
	\frac{1}{(1-z_{p_1}z^{-1})(1-z_{p_2}z^{-1})}
	\nonumber \\
 &=& \frac{1}{(1-r e^{j\theta}z^{-1})(1-r e^{-j\theta}z^{-1})} 
 = \frac{A}{1-r e^{j\theta}z^{-1}}+\frac{B}{1-r e^{-j\theta}z^{-1}}	
	\nonumber
\end{eqnarray}
where
\[
A=\frac{e^{j\theta}}{2j\;sin\theta},\;\;\;\;B=\frac{-e^{-j\theta}}{2j\;sin\theta}
\]
The impulse response of the system can be found by inverse z-transform 
\begin{eqnarray}
h[n]&=& [A(re^{j\theta})^n+B(re^{-j\theta})^n]u[n]
	=\frac{r^n}{2j\;sin\theta}[e^{j\theta(n+1)}-e^{-j\theta(n+1)}]u[n]
	\nonumber \\
 &=& r^n\frac{sin[(n+1)\theta]}{sin\theta}u[n]
	\nonumber 
\end{eqnarray}
When $\theta=0$, $H(z)$ has two repeated poles $z_p=r$ and becomes
\[ H(z)=\frac{1}{(1-rz^{-1})^2}	\]
and the impulse response becomes
\[	h[n]=(n+1)r^n u[n]	\]
When $\theta=\pi$, $H(z)$ has two repeated poles $z_p=-r$ and becomes
\[ H(z)=\frac{1}{(1+rz^{-1})^2}	\]
and the impulse response becomes
\[	h[n]=(n+1)(-r)^n u[n]	\]
To graphically evaluate the behavior of the system as a function of frequency 
$\omega$ in the z-plane, we let $z=e^{j\omega}$ in $H(z)$ and get the frequency 
response function of the system
\[
H(e^{j\omega})=\frac{1}{(1-re^{j\theta}e^{-j\omega})(1-re^{-j\theta}e^{-j\omega})}
=\frac{(e^{j\omega})^2}{(e^{j\omega}-re^{j\theta})(e^{j\omega}-re^{-j\theta})}	
=\frac{u^2}{v_1 \; v_2}	\]
where $u$, $v_1$ and $v_2$ are three vectors in z-plane defined as
\[ 	u\stackrel{\triangle}{=}e^{j\omega}-0=e^{j\omega},
\;\;\;\;v_1\stackrel{\triangle}{=}e^{j\omega}-re^{j\theta},
\;\;\;\;v_2\stackrel{\triangle}{=}e^{j\omega}-re^{-j\theta}
\]
For any frequency $-\pi \le\omega\le\pi$ represented by a point $z=e^{j\omega}$
on the unit circle, the magnitude and phase angle of the frequency response
function can be represented in the z-plane as
\[	|H(e^{j\omega})|=\frac{|u|^2}{|v_1||v_2|}	\]
and
\[	\angle{H(e^{j\omega})}=2\angle u-\angle v_1-\angle v_2	\]
When $\omega=\pm \pi$, $v_1$ and $v_2$ are maximized and thereby 
$|H(e^{j\omega})|$ is minimized. In particular, when $r$ is close to 1, 
$|v_1|=1-|r|$ is minimized when $\omega=\theta$, i.e., $|H(e^{j\omega})|$ 
has a peak.

\newpage
\section*{System Algebra and Block Diagram}

Z-transform converts time-domain operations such as difference and convolution
into algebraic operations in z-domain. Moreover, the behavior of complex systems
composed of a set of interconnected LTI systems can also be easily analyzed in 
z-domain. Some simple interconnections of LTI systems are listed below.
\begin{itemize}
\item {\bf Parallel systems: } If the system is composed of two LTI systems
with $h_1[n]$ and $h_2[n]$ connected in parallel, its impulse response is
\[ h[n]=h_1[n]+h_2[n]	\]
or in s-domain
\[ H(z)=H_1(z)+H_2(z)	\]

\item {\bf Serial or cascade system: } If the system is composed of two LTI 
systems with $h_1[n]$ and $h_2[n]$ connected in series, its impulse response is
\[ h[n]=h_1[n]*h_2[n]=h_2[n]*h_1[n]	\]
or in s-domain
\[ H(z)=H_1(z)H_2(z)=H_2(z)H_1(z)	\]

\htmladdimg{../Zdiagram0.gif}

\item {\bf Feedback system: } If the system is composed of an LTI system
with $h_1[n]$ in a forward path and another LTI system $h_2[n]$ in a 
feedback path, its output $y[n]$ can be implicitly found in time domain 
\[ y[n]=h_1[n]*e[n]=h_1[n]*[x[n]+h_2[n]*y[n]]	\]
or in s-domain 
\[ Y(z)=H_1(z)E(z)=H_1(z)[X(z)+H_2(z)Y(z)]	\]
While it is difficult to solve the equation in time domain to find an
explicit expression for output $y[n]=h[n]*x[n]$, it is easy to solve the
algebraic equation in z-domain to find $Y(z)$
\[	Y(z)[1-H_1(z)H_2(z)]=H_1(z) X(z)	\]
and the transfer function can be obtained
\[	H(z)=\frac{Y(z)}{X(z)}=\frac{H_1(z)}{1-H_1(z)H_2(z)}	\]
The feedback could be either positive or negative. For the latter, there will 
be a negative sign in front of $h_2(t)$ and $H_2(s)$ of the feedback path so that
$e[n]=x[n]-h_2[n]*y[n]$ and 
\[	H(z)=\frac{Y(z)}{X(z)}=\frac{H_1(z)}{1+H_1(z)H_2(z)}	\]
\end{itemize}

{\bf Example 0: } The transfer function of a first order LTI system
\[ y[n]-\frac{1}{4}y[n-1]=x[n],\;\;\;\;\;Y(z)[1-\frac{1}{4}z^{-1}]=X(z)	\]
is
\[	H(z)=\frac{1}{1-\frac{1}{4}z^{-1}}	\]
Comparing this $H(z)$ with the transfer function of the feedback system, we
see that a first order system can be represented as a feedback system with
$H_1(z)=1$ in the forward path, and $H_2(z)$ for the product of $1/4$ and 
$z^{-1}$ (a delay element with input $x[n]$ and output $y[n]=x[n-1]$) in the
negative feedback path.

\htmladdimg{../Zdiagram1.gif}

{\bf Example 1: } 
\[ H(z)=\frac{Y(z)}{X(z)}=\frac{1-2z^{-1}}{1-\frac{1}{4}z^{-1}} \]
This equation can be rewritten as:
\[ Y(z)=(1-2z^{-1}) \frac{1}{1-\frac{1}{4}z^{-1}}X(z)=(1-2z^{-1})W(z) \]
where 
\[ W(z)=\frac{1}{1-\frac{1}{4}z^{-1}}X(z) \]
can be obtained the same way as in previous example. Once $W(z)$ and
$W(z)z^{-1}$ are available, we can easily obtain $Y(z)$:

\htmladdimg{../Zdiagram2.gif}

{\bf Example 2: } Consider a second order system with transfer function
\[
H(z)=\frac{1}{1+\frac{1}{4}z^{-1}-\frac{1}{8}z^{-2}}
	=\frac{1}{(1+\frac{1}{2}z^{-1})(1-\frac{1}{4}z^{-1})}
	=\frac{2/3}{1+\frac{1}{2}z^{-1}}+\frac{1/3}{1-\frac{1}{4}z^{-1}}
\]
These three expressions of this $H(z)$ correspond to three different block
diagram representations of the system. The last two expressions are, 
respectively, the cascade and the parallel representations (same as the
corresponding cases in Laplace transform), while the first one is the 
direct representation, as shown below. We first consider a general 2nd order
system
\[	Y(z)+asY(z)z^{-1}+bY(z)z^{-2}=X(z),\;\;\;\;\;\mbox{or}\;\;\;\;\;\;
	Y(z)=X(z)-aY(z)z^{-1}-bY(z)z^{-2}	\]
We see that $Y(z)$ is the linear combination of the delayed versions of 
itself and the input $X(z)$ which can be represented as a feedback system with
two feedback paths of $-az^{-1}$ and $-bz^{-2}$. In this particular system,
$a=-1/4$ and $b=1/8$.

\htmladdimg{../Zdiagram3.gif}

\htmladdimg{../Zdiagram4.gif}

\htmladdimg{../Zdiagram5.gif}


{\bf Example 3: } A second order system with transfer function
\[	H(z)=K \frac{1+cz^{-1}+dz^{-2}}{1+az^{-1}+bz^{-2}}
	=\frac{K}{1+az^{-1}+bz^{-2}}(1+cz^{-1}+dz^{-2})
\]
This system can be represented as a cascade of two systems
\[	W(z)=H_1(z)X(z)=\frac{K}{1+az^{-1}+bz^{-2}}X(z)	\]
and 
\[	Y(z)=H_2(z)W(z)=(1+cz^{-1}+dz^{-2})W(z)	\]
The first system $H_1(z)$ can be implemented by two delay elements with proper 
feedback paths as shown in the previous example, and the second system is a 
linear combination of $W(z)$, $W(z)z^{-1}$ and $W(z)z^{-2}$, all of which are
available along the feedback path of the first system. The over all system can
therefore be represented as shown. Obviously the block diagram of this example 
can be generalized to represent any system with a rational transfer function
\[
H(z)=\frac{\sum_{k=0}^M b_k z^{-k}}{\sum_{k=0}^N a_k z^{-k}}\;\;\;\;(M \le N)	
\]

\htmladdimg{../Zdiagram6.gif}

\section*{Unilateral Z-Transform}

The {\em unilateral} z-transform of an arbitrary signal $x[n]$ is defined as
\[	{\cal UZ}[x[n]]=X(z)\stackrel{\triangle}{=}
\sum_{n=-\infty}^\infty x[n]u[n] z^{-n}=\sum_{n=0}^\infty x[n] z^{-n}
\]
When the unilateral z-transform is applied to find the transfer function 
$H(z)={\cal UZ}[h[n]]$ of an LTI system, it is always assumed to be causal, 
and the ROC is always the exterior of a circle. The unilateral z-transform
of any signal $x[n]=x[n]u[n]$ is identical to its bilateral Laplace transform.
However, if $x[n] \ne x[n]u[n]$, the two z-transforms are different. Some of 
the properties of the unilateral z-transform different from the bilateral 
z-transform are listed below.

\begin{itemize}
\item {\bf Time Advance}
\begin{eqnarray}
  {\cal UZ}\left[x[n+1]\right]&=&\sum_{n=0}^\infty x[n+1]z^{-n}
  =z\sum_{m=1}^\infty x[m]z^{-m}
  \nonumber \\
  &=&z\left[\sum_{m=0}^\infty x[m]z^{-m}-x[0]\right]=zX(z)-zx[0] 
  \nonumber \end{eqnarray}
where we have assumed $m=n+1$.

\item {\bf Time Delay}
\begin{eqnarray}
  {\cal UZ}\left[x[n-1]\right]&=&\sum_{n=0}^\infty x[n-1]z^{-n}
  =z^{-1}\sum_{m=-1}^\infty x[m]z^{-m}
  \nonumber \\
  &=& z^{-1}\left[ \sum_{m=0}^\infty x[m]z^{-m}+zx[-1]\right]=z^{-1}X(z)+x[-1]
  \nonumber
\end{eqnarray}
where $m=n-1$. Similarly, we have
\begin{eqnarray}
  {\cal UZ}[x[n-2]] &=& \sum_{n=0}^\infty x[n-2]z^{-n}
  =z^{-2}\sum_{m=-2}^\infty x[m]z^{-m}
  \nonumber \\
  &=& z^{-2}\left[\sum_{m=0}^\infty x[m]z^{-m}+zx[-1]+z^2 x[-2]\right]
  \nonumber \\
  &=&z^{-2}X(z)+x[-1]z^{-1}+x[-2]
  \nonumber
\end{eqnarray}
where $m=n-2$. In general, we have
\[ {\cal UZ}[x[n-n_0]] = z^{-n_0}X(z)+\sum_{k=0}^{n_0-1} z^{-k} x[k-n_0] \]

\item {\bf Convolution}
\[	{\cal UZ}[x[n]*y[n]]=X(z)Y(z)	\]
If both $x[n]$ and $y[n]$ are causal, i.e., $x[n]=y[n]=0$ for $n<0$, the
unilateral and bilateral z-transforms are identical.

\item {\bf Time Difference}
\[	{\cal UZ}[x[n]-x[n-1]]=(1-z^{-1})X(z) \;\;\;\;\;ROC=R_x  	\]
{\bf Proof: } 
\[	
{\cal UZ}[x[n]-x[n-1]]=X(z)-z^{-1}X(z)-x[-1]=(1-z^{-1})X(z)-x[-1]
\]

\item {\bf Time Accumulation}
\[	{\cal UZ}[\sum_{k=0}^nx[k]]=\frac{1}{1-z^{-1}}X(z)	\]

\item {\bf Initial Value Theorem}

If $x[n]=x[n]u[n]$, i.e., $x[n]=0$ for $n<0$, then
\[	x[0]=\lim_{z\rightarrow \infty} X(z)	\]
{\bf Proof: }
\[	\lim_{z\rightarrow \infty} X(z)=\lim_{z\rightarrow \infty} 
	\left[\sum_{n=0}^\infty x[n]z^{-n}\right]=x[0]
\]
All terms with $n>0$ become zero as $z^{-n}=1/z^n \rightarrow 0$ as 
$z \rightarrow \infty$, except the first one which is always $x[0]$.

\item {\bf Final Value Theorem}

If $x[n]=x[n]u[n]$, i.e., $x[n]=0$ for $n<0$, then
\[ \lim_{n\rightarrow \infty}x[n]=x[\infty]
=\lim_{z\rightarrow 1}(1-z^{-1}) X(z)	\]
{\bf Proof: } 
\[ {\cal Z}\left[ x[n]-x[n-1]\right]=X(z)-X(z)z^{-1}
=\sum_{n=0}^\infty \left[ x[n]-x[n-1] \right]z^{-n} \]
i.e.
\[ (1-z^{-1})X(z)=\lim_{N\rightarrow \infty}\sum_{n=0}^N \left[ x[n]-x[n-1]\right]z^{-n}\]
Letting $z\rightarrow 1$ in the above, we get
\begin{eqnarray}
  \lim_{z\rightarrow 1}(1-z^{-1})X(z)
  &=&\lim_{N\rightarrow \infty}\left[ \sum_{n=0}^N [x[n]-x[n-1]\right]
    \lim_{N\rightarrow \infty}\left[ x[N]-x[-1]\right]
  &=&\lim_{N\rightarrow \infty} x[N]=x[\infty]
  \nonumber
\end{eqnarray}
where $x[-1]=0$.

\end{itemize}

{\bf Example:}
\[	x[n]=a^{n+1}u[n+1] 	\]
This signal is right sided starting at $n=-1$ (i.e., $x[n] \ne x[n]u[n]$).
By definition, the bilateral z-transform of $x[n]$ is
\[	{\cal Z}[x[n]]=\sum_{n=-1}^\infty a^{n+1}z^{-n}
	=z+a\sum_{n=0}^\infty a^n z^{-n}=z+\frac{a}{1-az^{-1}}
	=\frac{z}{1-az^{-1}}
\]
It was assumed that $|z|>|a|$. The unilateral z-transform of this signal is
\[ {\cal UZ}[x[n]]=\sum_{n=0}^\infty a^{n+1}z^{-n}
	=a\sum_{n=0}^\infty a^n z^{-n}=\frac{a}{1-az^{-1}}
\]

If we assume zero initial condition $y[-1]=0$, 

\newpage
\section*{Solving LCCDEs by Unilateral Z-Transform}

Due to its time delay property, the unilateral z-transform is a powerful 
tool for solving LCCDEs with arbitrary initial conditions. 

{\bf Example: } A system is described by this LCCDE
\[	y[n]+3y[n-1]=x[n]=\alpha u[n]	\]
Taking unilateral z-transform of the DE, we get
\[	Y(z)+3Y(z)z^{-1}+3y[-1]=X(z)=\frac{\alpha}{1-z^{-1}}	\]
\begin{itemize}
\item {\bf The particular (zero-state) solution}

If the system is initially at rest, i.e., $y[-1]=0$, the above equation can 
be solved for the output $Y(z)$ to get
\[	Y(z)=H(z)X(z)=\frac{1}{1+3z^{-1}}\;\frac{\alpha}{1-z^{-1}}
	=\frac{3\alpha/4}{1+3z^{-1}}+\frac{\alpha/4}{1-z^{-1}}	\]	
where $H(z)=1/(1+3z^{-1})$ is the system's transfer function. In time domain
this is the particular (or zero-state) solution (caused by the input with 
zero initial condition):
\[ y_p[n]=\alpha[\frac{1}{4}+\frac{3}{4}(-3)^n]u[n]	\]

\item {\bf The homogeneous (zero-input) solution}

When the initial condition is nonzero 
\[	y[-1]=\beta	\]
but the input is zero $x[n]=0$, the z-transform of the difference equation 
becomes
\[	Y(z)+3Y(z)z^{-1}+3\beta=0	\]
which can be solved for $Y(z)$ 
\[	Y(z)=\frac{-3\beta}{1+3z^{-1}}	\]
In time domain, this is the homogeneous (or zero-input) solution (caused by the
initial condition with zero input):
\[	y_h[n]=-3\beta (-3)^n u[n]	\]
\end{itemize}
When neither $y[-1]$ nor $x[n]$ is zero, we have
\[	Y(z)+3Y(z)z^{-1}+3\beta=X(z)=\frac{\alpha}{1-z^{-1}}	\]
Solving this algebraic equation in z-domain for $Y(z)$ we get
\[	Y(z)=\frac{\alpha}{(1+3z^{-1})(1-z^{-1})}-\frac{3\beta}{1+3z^{-1}} \]
The first term is the particular solution caused by the input alone and the 
second term is the homogeneous solution caused by the initial condition alone.
The $Y(z)$ can be further written as
\[	Y(z)=\frac{1}{1+3z^{-1}}(\frac{3}{4}\alpha-3\beta)
	+\frac{\alpha}{4}\frac{1}{1-z^{-1}}
\]
and in time domain, we have the general solution
\[	y_g[n]=[(\frac{3}{4}\alpha-3\beta)(-3)^n+\frac{\alpha}{4}]u[n]
	=y_h[n]+y_p[n]	\]
which is the sum of both the homogeneous and particular solutions.

Note that bilateral z-transform can also be used to solve LCCDEs. However,
as bilateral z-transform does not take initial condition into account, it is
always implicitly assumed that the system is initially at rest. If this is not 
the case, unilateral z-transform has to be used.
%}
\end{document}

\[	\sum_{n=0}^\infty a^n=\frac{1}{1-a}\;\;\;\;\mbox{for $|a|<1$} \]
{\bf Proof:}
First consider
\[	S_N\stackrel{\triangle}{=}\sum_{n=0}^N a^n=1+a+a^2+\cdots+a^N \]
multiplying both sides by $a$, we get
\[	aS_N=a+a^2+\cdots+a^N+a^{N+1} \]
subtracting the second expression from the first, we have
\[	(1-a)S_N=1-a^{N+1}	\]
or
\[	S_N=\frac{1-a^{N+1}}{1-a}	\]
If $|a|<1$, $\lim_{n\rightarrow \infty}a^{N+1}=0$, and
\[	\lim_{n\rightarrow \infty}S_N=\sum_{n=0}^\infty a^n=\frac{1}{1-a} \]



\begin{itemize}
\item {\bf Right sided:}
\[	x[n]=a^n u[n],\;\;\;\;\;(|a|>1)	\]
Its discrete Fourier transform does not exist as the signal grows without
a bound when $n\rightarrow \infty$, i.e., the transform summation does not 
converge. However, its z-transform $X(z)$ exists if $|z|=e^\sigma>|a|$ 
(i.e., $|a/z|<1$), as the modified signal 
$a^n e^{-\sigma n} u[n]=(a/e^\sigma)^n u[n]$ will converge when 
$n\rightarrow -\infty$.

\item {\bf Left sided:}
\[	x[n]=a^n u[-n],\;\;\;\;\;(|a|<1)	\]
Again the discrete Fourier transform does not exist as the signal grows when
$n\rightarrow -\infty$, i.e., the transform summation does not converge (not
summable). However, its z-transform exists if $|z|=e^\sigma<|a|$ (i.e., 
$|a/z|>1$), as the modified signal 
$a^n e^{-\sigma n} u[-n]=(a/e^\sigma)^n u[-n]$ will converge when 
$n\rightarrow -\infty$.
\end{itemize}

\newpage
