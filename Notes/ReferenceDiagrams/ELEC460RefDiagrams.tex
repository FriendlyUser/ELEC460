
\documentclass[a4paper,11pt]{report}
%\usepackage[
%HomeHTMLFilename=index,     % Filename of the homepage.
%%HTMLFilename={node-},       % Filename prefix of other pages.
%IndexLanguage=english,      % Language for xindy index, glossary.
%latexmk,                    % Use latexmk to compile.
%%   OSWindows,                  % Force Windows. (Usually automatic.)
%mathjax,                    % Use MathJax to display math.
%]{lwarp}
%\title{ELEC 460 Some Useful Notes}
%\author{David Li}
%\setcounter{tocdepth}{2} % Include subsections in the \TOC.
%\setcounter{secnumdepth}{2} % Number down to subsections.
%\setcounter{FileDepth}{0} % Split \HTML\ files at sections, in this case chapters?, 0 for chapters?
%\booltrue{CombineHigherDepths} % Combine parts/chapters/sections
%\setcounter{SideTOCDepth}{1} % Include subsections in the side\TOC
%\HTMLAuthor{David Li} % Sets the HTML meta author tag.
%\HTMLLanguage{en-US} % Sets the HTML meta language.
%\HTMLDescription{A list of cheatsheets some diagrams}%

\usepackage{tikz}
\usetikzlibrary{positioning}
\usetikzlibrary{shapes,arrows} 
\newcommand{\sse}{\mathrm{ss}}
\newcommand{\re}{\mathrm{ref}}
\usepackage{float}
\usepackage{amsmath}
%\usepackage{accents}
\usepackage{amssymb}
\begin{document}
	
\section{Steady-state error}

\subsection{Definition}

Consider a feedback system as in Fig.~\ref{fig:sse}.

\begin{figure}[h]
	\centering
	\tikzstyle{block} = [draw, fill=blue!20, rectangle, minimum height=3em, minimum width=4em]
	\tikzstyle{block2} = [draw, fill=blue!20, rectangle, minimum height=2em, minimum width=2em]
	\tikzstyle{sum} = [draw, fill=blue!20, circle, node distance=1cm]
	\tikzstyle{input} = [coordinate]
	\tikzstyle{output} = [coordinate]
	\begin{tikzpicture}[auto, >=latex']
	% Nodes
	\node [input] (input) {};
	\node [sum, right = 1cm of input] (sum) {};
	%\node [block2, right = 1cm of sum] (system) {$K$};
	\node [block, right = 1cm of sum] (system2) {$G$};
	\node [output, right = 2cm of system2] (output) {};
	\node [block, below = 0.5cm of system2] (h) {$H$};
	% Arrows
	\draw [draw,->] (input) -- node {$R$} (sum);
	\draw [->] (sum) -- node {$E$} (system2);
	%\draw [->] (system) -- (system2);
	\draw [->] (system2) -- node (y) {$C$}(output);
	\draw [-] (y) |- (h) {} ;
	\draw [->] (h) -| node[pos=0.99] {$-$}  node [near end] {} (sum);
	\end{tikzpicture}  
	\caption{Error signal: $E=R-C$}
	\label{fig:sse}
\end{figure}

The steady-state error (if it exists) is defined by
\begin{equation}
	\label{eq:sse}
	e_\sse := \lim_{t\to\infty} e(t).
\end{equation}

If the system is stable, one can apply the Finaly Value Theorem to obtain
\[
\lim_{t\to\infty} e(t) = \lim_{s\to 0} sE(s).
\]

In general, one would like the steady-state error to be as small as
possible, ideally zero.

\subsection{System type and steady-state error}

In general, to calculate the steady-state error, one uses the FVT as
in equation~(\ref{eq:sse}). However, if the system is stable and
\emph{unity-feedback}, then one can determine the system type and use
this information to obtain the steady-state error much more easily.

Let $G$ be the feedforward transfer function. Assume that $G$ is
written as
\[
G=\frac{N(s)}{s^kD(s)},
\]
where $N$ and $D$ are not factorizable by $s$. Then the type of the
system is given by the integer $k$. Fig.~\ref{fig:systype} shows the
general form of systems of types 1, 2 and 3.

\begin{figure}[h]
	\hspace{1.5cm}\textbf{Type 0}\hspace{3cm}\textbf{Type
		1}\hspace{3cm}\textbf{Type 2}\\
	\vspace{0.1cm} 
	
	\centering \tikzstyle{block} = [draw, fill=blue!20,
	rectangle, minimum height=3em, minimum width=3em]
	\tikzstyle{controller} = [draw, fill=red!20, rectangle, minimum
	height=3em, minimum width=4em] \tikzstyle{sum} = [draw,
	fill=blue!20, circle, node distance=1cm] \tikzstyle{input} =
	[coordinate] \tikzstyle{output} = [coordinate]
	\begin{tikzpicture}[auto, >=latex']
	% Nodes
	\node [input] (input) {};
	\node [sum, right = 0.5cm of input] (sum) {};
	\node [block, right = 0.5cm of sum] (system) {$\frac{N_0(s)}{D_0(s)}$};
	\node [output, right = 1cm of system] (output) {};
	\node [input, below = 0.5cm of system] (m) {};
	% Arrows
	\draw [draw,->] (input) -- node {$R$} (sum);
	\draw [->] (sum) -- node {} (system);
	\draw [->] (system) -- node (y) {$C$}(output);
	\draw [-] (y) |- (m) {} ;
	\draw [->] (m) -| node[pos=0.99] {$-$}  node [near end] {} (sum);
	\end{tikzpicture}  
	\hspace{0.5cm}
	\begin{tikzpicture}[auto, >=latex']
	% Nodes
	\node [input] (input) {};
	\node [sum, right = 0.5cm of input] (sum) {};
	\node [block, right = 0.5cm of sum] (system) {$\frac{N_1(s)}{sD_1(s)}$};
	\node [output, right = 1cm of system] (output) {};
	\node [input, below = 0.5cm of system] (m) {};
	% Arrows
	\draw [draw,->] (input) -- node {$R$} (sum);
	\draw [->] (sum) -- node {} (system);
	\draw [->] (system) -- node (y) {$C$}(output);
	\draw [-] (y) |- (m) {} ;
	\draw [->] (m) -| node[pos=0.99] {$-$}  node [near end] {} (sum);
	\end{tikzpicture}  
	\hspace{0.5cm}
	\begin{tikzpicture}[auto, >=latex']
	% Nodes
	\node [input] (input) {};
	\node [sum, right = 0.5cm of input] (sum) {};
	\node [block, right = 0.5cm of sum] (system) {$\frac{N_2(s)}{s^2D_2(s)}$};
	\node [output, right = 1cm of system] (output) {};
	\node [input, below = 0.5cm of system] (m) {};
	% Arrows
	\draw [draw,->] (input) -- node {$R$} (sum);
	\draw [->] (sum) -- node {} (system);
	\draw [->] (system) -- node (y) {$C$}(output);
	\draw [-] (y) |- (m) {} ;
	\draw [->] (m) -| node[pos=0.99] {$-$}  node [near end] {} (sum);
	\end{tikzpicture}  
	\caption{Systems of types 0, 1 and 2. Note that $N_0, N_1, N_2, D_0,
		D_1, D_2$ should not be factorizable by $s$.}
	\label{fig:systype}
\end{figure}

Let us carry further the computation of the steady-state error for a
stable unity-feedback system.
\[
sE(s) = s(R(s)-C(s)) = s\left(R(s)-\frac{G(s)}{1+G(s)}R(s)\right)=\frac{s}{1+G(s)}R(s).
\]

For unit-step input, one has $R(s)=1/s$. Thus,
\[
sE(s) = \frac{s}{1+G(s)}*\frac{1}{s} = \frac{1}{1+G(s)}
\xrightarrow[s\to 0]{}\frac{1}{1+\lim_{s\to 0}G(s)}.
\]
For a system of type 0, one has
\[
\lim_{s\to 0}G(s) = \lim_{s\to 0}\frac{N_0(s)}{D_0(s)} = \frac{N_0(0)}{D_0(0)}.
\]
Thus, the steady-state error of a system of type 0 for unit-step input
is given by
\[
e_\sse = \frac{1}{1+\frac{N_0(0)}{D_0(0)}}.
\]

Carrying out similar calculations, one obtains the following table,
which shows the steady-state errors of systems of types 0, 1, 2, 3 for
unit-step, unit-ramp and unit-parabola inputs.

\begin{table}[htp]
	\centering
	\begin{tabular}{|c|c|c|c|}
		\hline
		System type&Step input&Ramp input&Parabolic input\\\hline
		Type 0&$\frac{1}{1+K_p}$&$\infty$&$\infty$\\\hline
		Type 1&0&$\frac{1}{K_v}$&$\infty$\\\hline
		Type 2&0&0&$\frac{1}{K_a}$\\\hline
		Type 3&0&0&0\\\hline
	\end{tabular}
	\caption{Steady-state errors of systems of different types and for
		different inputs.}
	\label{tab:systype}
\end{table}

The constants $K_p$, $K_v$, and $K_a$ are defined as below
\begin{itemize}
	\item for a system of type 0, $K_p:=\frac{N_0(0)}{D_0(0)}$ (position constant);
	\item for a system of type 1, $K_v:=\frac{N_1(0)}{D_1(0)}$ (velocity constant);
	\item for a system of type 2, $K_a:=\frac{N_2(0)}{D_2(0)}$
	(acceleration constant).
\end{itemize}

\subsection{Other Block Diagrams}
In the Laplace domain, we have
\[
I = K_D(sZ_\re-sZ) + K_P(Z_\re-Z) = (K_Ds+K_P)(Z_\re-Z),
\]
which leads to the following block diagram.

\begin{figure}[H]
	\label{fig:examplesystempd}
	\centering
	\tikzstyle{block} = [draw, fill=blue!20, rectangle, minimum height=3em, minimum width=4em]
	\tikzstyle{controller} = [draw, fill=red!20, rectangle, minimum height=3em, minimum width=4em]
	\tikzstyle{sum} = [draw, fill=blue!20, circle, node distance=1cm]
	\tikzstyle{input} = [coordinate]
	\tikzstyle{output} = [coordinate]
	\begin{tikzpicture}[auto, >=latex']
	% Nodes
	\node [input] (input) {};
	\node [sum, right = 1cm of input] (sum) {};
	\node [controller, right = 1cm of sum] (system) {$K_D s + K_P$};
	\node [block, right = 1cm of system] (system2) {$\frac{k}{ms^2}$};
	\node [output, right = 2cm of system2] (output) {};
	\node [input, below = 0.5cm of system] (m) {};
	% Arrows
	\draw [draw,->] (input) -- node {$Z_\re$} (sum);
	\draw [->] (sum) -- node {} (system);
	\draw [->] (system) -- node {$I$} (system2);
	\draw [->] (system2) -- node (y) {$Z$}(output);
	\draw [-] (y) |- (m) {} ;
	\draw [->] (m) -| node[pos=0.99] {$-$}  node [near end] {} (sum);
	\end{tikzpicture}  
	\caption{Proportional-derivative control.}
\end{figure}

\begin{figure}[H]
	\label{fig:lag}
	\centering
	\tikzstyle{block} = [draw, fill=blue!20, rectangle, minimum height=3em, minimum width=4em]
	\tikzstyle{controller} = [draw, fill=red!20, rectangle, minimum height=3em, minimum width=4em]
	\tikzstyle{sum} = [draw, fill=blue!20, circle, node distance=1cm]
	\tikzstyle{input} = [coordinate]
	\tikzstyle{output} = [coordinate]
	\begin{tikzpicture}[auto, >=latex']
	% Nodes
	\node [input] (input) {};
	\node [sum, right = 1cm of input] (sum) {};
	\node [controller, right = 1cm of sum] (con1) {$\frac{s+p_c}{s+z_c}$};
	\node [controller, right = 1cm of con1] (con2) {$K_D s + K_P$};
	\node [block, right = 1cm of con2] (system2) {$\frac{k}{ms^2}$};
	\node [output, right = 2cm of system2] (output) {};
	\node [input, below = 0.5cm of con2] (m) {};
	% Arrows
	\draw [draw,->] (input) -- node {$Z_\re$} (sum);
	\draw [->] (sum) -- (con1);
	\draw [->] (con1) -- (con2);
	\draw [->] (con2) -- node {$I$} (system2);
	\draw [->] (system2) -- node (y) {$Z$}(output);
	\draw [-] (y) |- (m) {} ;
	\draw [->] (m) -| node[pos=0.99] {$-$}  node [near end] {} (sum);
	\end{tikzpicture}  
	\caption{Lag compensation in series with PD control.}
\end{figure}

\begin{figure}[H]
	\label{fig:integral}
	\centering
	\tikzstyle{block} = [draw, fill=blue!20, rectangle, minimum height=3em, minimum width=4em]
	\tikzstyle{controller} = [draw, fill=red!20, rectangle, minimum height=3em, minimum width=4em]
	\tikzstyle{sum} = [draw, fill=blue!20, circle, node distance=1cm]
	\tikzstyle{input} = [coordinate]
	\tikzstyle{output} = [coordinate]
	\begin{tikzpicture}[auto, >=latex']
	% Nodes
	\node [input] (input) {};
	\node [sum, right = 1cm of input] (sum) {};
	\node [controller, right = 1cm of sum] (system) {$\frac{K}{s}$};
	\node [block, right = 1cm of system] (system2) {$\frac{1}{Ts+1}$};
	\node [output, right = 2cm of system2] (output) {};
	\node [input, below = 0.5cm of system] (m) {};
	% Arrows
	\draw [draw,->] (input) -- node {$R$} (sum);
	\draw [->] (sum) -- node {} (system);
	\draw [->] (system) --  (system2);
	\draw [->] (system2) -- node (y) {$C$}(output);
	\draw [-] (y) |- (m) {} ;
	\draw [->] (m) -| node[pos=0.99] {$-$}  node [near end] {} (sum);
	\end{tikzpicture}  
	\caption{First order system with integral controller.}
\end{figure}
\end{document}