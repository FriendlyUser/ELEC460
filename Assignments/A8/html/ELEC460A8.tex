
%% Creator David Li
% Modified matlab xsl latex template file to suit needs
% This LaTeX was auto-generated from an M-file by MATLAB.
% To make changes, update the M-file and republish this document.

\documentclass[12pt]{scrartcl}
\nonstopmode

\title{Matlab}
\usepackage[utf8]{inputenc}
\usepackage{graphicx}
\usepackage{epstopdf}
\usepackage{color}
\usepackage{xcolor}
\usepackage{amsmath}
\usepackage[ocgcolorlinks]{hyperref}
\usepackage{bookmark}
\usepackage[hmargin=2cm,vmargin=2.5cm]{geometry}
\usepackage{fancyhdr}
\usepackage{booktabs}
\sloppy
\definecolor{lightgray}{gray}{0.5}
% \definecolor{myText}{HTML}{2B2B2B}
\definecolor{fontColor}{HTML}{171717}
\setlength{\parindent}{0pt}

\makeindex

\usepackage{listings}
\definecolor{mygreen}{RGB}{28,172,0} % color values Red, Green, Blue
\definecolor{mylilas}{RGB}{170,55,241}
\lstset{language=Matlab,%
	%basicstyle=\color{red},
	breaklines=true,%
	morekeywords={matlab2tikz},
	keywordstyle=\color{blue},%
	morekeywords=[2]{1}, keywordstyle=[2]{\color{black}},
	identifierstyle=\color{black},%
	stringstyle=\color{mylilas},
	commentstyle=\color{mygreen},%
	showstringspaces=false,%without this there will be a symbol in the places where there is a space
	%numbers=left,%
	%numberstyle={\tiny \color{black}},% size of the numbers
	%numbersep=9pt, % this defines how far the numbers are from the text
	emph=[1]{for,end,break},emphstyle=[1]\color{red}, %some words to emphasise
	%emph=[2]{word1,word2}, emphstyle=[2]{style},  
    captionpos=b,
    caption={Matlab Code Snippet:},
}
\usepackage{tcolorbox}
\tcbuselibrary{listings}
\tcbuselibrary{breakable}


\newtcblisting[auto counter,number within=section*]{matlaboutput}[2][]{sharp corners, breakable,
    fonttitle=\bfseries,colback=white, colframe=black!90, listing only, 
    listing options={language=Matlab, showstringspaces=false, breakatwhitespace=true, breaklines=true, tabsize=4}, 
    title=Matlab Output \thetcbcounter: #1} 

\usepackage{fancyhdr} 
\fancyhf{}
\cfoot{\thepage}
\pagestyle{fancy}

\begin{document}

\begin{center}
	\hrule
	\vspace{.4cm}
	{\textbf { \large ELEC 460 --- Applied Electromagnetics and Photonics}}
\end{center}
{\textbf{Name:}\ David Li \hspace{\fill} \textbf{Student Number:} \ V00818631  \\
{\textbf{Due Date:} Thursday, 11 January 2018, 11:30 AM \hspace{\fill} \textbf{Assignment}  1}\\
\hrule

    
    
\section*{ELEC 460 -- Control Theory II}
      
\tableofcontents
\newpage


\subsection*{B-6-11}



        \begin{par}
Finding the rank of the controlability matrix
\end{par} \vspace{1em}
\begin{lstlisting}[language = Matlab,frame=single,caption={}]
syms T
G = [1 T; 0 1]; H = [T^2/2; T]
test = [H G*H]
\end{lstlisting}

 \begin{matlaboutput}{} 
H =
 
 T^2/2
     T
 
 
test =
 
[ T^2/2, (3*T^2)/2]
[     T,         T]
 
\end{matlaboutput}
    \begin{lstlisting}[language = Matlab,frame=single,caption={}]
syms u1 u2
J = [u1 0; ...
     0  u2;]
% p =  CHARPOLY(J)
\end{lstlisting}

 \begin{matlaboutput}{} 
J =
 
[ u1,  0]
[  0, u2]
 
\end{matlaboutput}
    

\subsection*{B-6-12}



        \begin{par}
Last question quick check
\end{par} \vspace{1em}
\begin{lstlisting}[language = Matlab,frame=single,caption={}]
u1 = 0.787592+j*0.238241
u2 = 0.787592-j*0.238241
T  = 0.1
K = [1/T^2*(1-u1-u2+u1*u2) 1/(2*T)*(3-u1-u2-u1*u2)]
% testing = zpk([],[],1), can't figure out how to plot to verify, need open
% loop transfer function
\end{lstlisting}

 \begin{matlaboutput}{}
u1 =

   0.7876 + 0.2382i


u2 =

   0.7876 - 0.2382i


T =

    0.1000


K =

   10.1876    3.7388

\end{matlaboutput}
    


\end{document}
    
